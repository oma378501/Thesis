\chapter{FORTALECIMIENTO DEL APRENDIZAJE AUTORREGULADO EN CARRERA LAEL}

\section{Objeto Formativo}

Formar profesionales para comprometerse con su medio con agente de cambio, 
interpretar la realidad educativa nacional, particularmente la ling\"{u}ista,
proponiendo metodolog\'{i}as espec\'{i}ficas para la ense\~{n}anza de lenguas
extranjeras, del castellano y el quechua, lengua extranjera y/o segunda.

\section{Perfil Profesional}

El profesional en LAEL, esta capacitado(a) para:

\begin{itemize}
\item Estar comprometido con su medio en el que gracias a procesos de 
investigaci\'{o}n de la realidad boliviana, aplicar\'{a} m\'{e}todos
y t\'{e}cnicas adecuadas dentro del proceso de ense\~{n}anza aprendizaje
de lenguas en el Sitema Educativo Nacional y Universitario.

\item Evaluar y adaptar m\'{e}todos de ense\~{n}anza tanto para las lenguas
extranjeras como para el castellano y el quechua: lengua extranjera y/o 
segunda lengua.

\item Utilizar el patrimonio literario-cultural de las lenguas estudiadas,
para reflejar sus valores est\'{e}ticos y humanos.

\item Realizar estudios e interpretaciones en el \'{a}rea de investigaci\'{o}n
interdiciplinaria ya sea en ense\~{n}anza de lenguas en el Sistema educativo,
problemas de alfabetizaci\'{o}n en nuestro pa\'{i}s, a nivel de aporte desde
la perspectiva de las lenguas, caracter\'{i}sticas del castellano boliviano
en sus distintos niveles culturales.

\item Realizar investigaciones sobre la lengua haciendo estudios comparativos
de sistemas de comunicaci\'{o}n, observando el funcionamiento y estructuras de
las lenguas, posibilidades de la implementaci\'{o}n de metodolog\'{i}as nuevas,
propias y adecuadas.

\item Evaluar los contenidos de la asignaturas relacionadas con el \'{a}rea de
lenguas en todos los niveles de ense\~{n}anza.

\item Desempe\~{n}ar eficientemente en cualquier otro campo en el que se exija
el conocimiento y formaci\'{o}n en lenguas.

\end{itemize}

\section{Objetivos Generales}

El licenciado en LAEL ser\'{a} capaz de:

\begin{itemize}

\item Desenvolverse en su medio como agente de cambio, comprometi\'{e}ndose 
profesionalmente con este.

\item Analizar la realidad educativa nacional, particularmente la ling\"{u}istica,
proponiendo metodolog\'{i}as espec\'{i}ficas para la ense\~{n}anza de lenguas( 
castellano, lengua nativa y/o extranjera).

\item Planificar la ense\~{n}anza de las lenguas extranjeras, del castellano y 
del quechua u otra lengua ind\'{i}gena, en los diferentes niveles de ense\~{n}anza
del Sistema Educativo Plurinacional: inicial, primario, secundario y universitario.

\item Evaluar, dise\~{n}ar y/o adaptar materiales did\'{a}cticos para la 
ense\~{n}anza de languas.

\end{itemize}

\section{Mercado Profesional}

Los profesionales egresados de la Carrera LAEL podr\'{a}n desempe\~{n}arse 
profesionalmente en las siguientes \'{a}reas:

\subsection{Ense\~{n}anza de Lenguas}

\begin{description}
\item[Educaci\'{o}n Primaria y Secundaria:] Ense\~{n}anza de ingl\'{e}s y 
franc\'{e}s como lengua extranjera; ense\~{n}anza de castellano y quechua
como lengua materna o seguna lengua.

\item[Educaci\'{o}n superior:] Ense\~{n}anza de ingl\'{e}s y franc\'{e}s como 
lenguas extranjeras; ense\~{n}anza de quechua como segunda lengua y ense\~{n}anza
de castellano como lengua materna.

\end{description}

\subsection{Dise\~{n}o y planificaci\'{o}n}

\begin{itemize}

\item Dise\~{n}o curricular.
\item Dise\~{n}o de planes y programas de estudio de lenguas, materna, segunda
y extranjera.
\item Dise\~{n}o de materiales did\'{a}cticos.
\item Adaptaci\'{o}n y adecuaci\'{o}n de materiales did\'{a}cticos.

\end{itemize}

\subsection{Evaluaci\'{o}n}

\begin{itemize}

\item Evaluaci\'{o}n de procesos de ense\~{n}anza y aprendizaje.
\item Evaluaci\'{o}n de materiales de ense\~{n}anza de lenguas.
\item Evaluaci\'{o}n de programas de ense\~{n}anza de lenguas.

\end{itemize}