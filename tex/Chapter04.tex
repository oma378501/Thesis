\chapter{FORTALECIMIENTO DEL APRENDIZAJE AUTORREGULADO EN CARRERA LAEL}

\section{Objeto Formativo}

Formar profesionales para comprometerse con su medio con agente de cambio, 
interpretar la realidad educativa nacional, particularmente la ling\"{u}\'{i}sta,
proponiendo metodolog\'{i}as espec\'{i}ficas para la ense\~{n}anza de lenguas
extranjeras, del castellano y el quechua, lengua extranjera y/o segunda.

\section{Perfil Profesional del Estudiante en la Carrera LAEL}

El estudiante en la Carrera LAEL el documento denominado \textquotedblleft 
Correspondencia y recibida y Despachada\textquotedblright del a\~{n}o 2009,
destac\'{o} el Perfil profesional del Estudiante de la Carrera LAEL constituido
textualmente de la siguiente manera:

\begin{enumerate}

\item Un profesional comprometido con su medio en el que gracias a procesos de
investigaci\'{o}n de la realidad boliviana aplicara m\'{e}todos y t\'{e}cnicas
adecuados dentro del proceso de ense\~{n}anza aprendizaje de lenguas en el 
sistema educativo nacional y universitario
\item Ser capaz de evaluar y adaptar m\'{e}todos de ense\~{n}anza, tanto para
las lenguas extranjeras como para el castellano y el Quechua: lengua extranjera
y/o segunda lengua.
\item Realizar investigaci\'{o}n interdisciplinaria para estudios e 
interpretaci\'{o}n sobre:
	\begin{itemize}
	
	\item La ense\~{n}anza de lenguas en el sistema educativo
	\item Problemas de alfabetizaci\'{o}n en nuestro pa\'{i}s, aportando desde la
	perspectiva de las lenguas
	\item Problemas espec\'{i}ficos de biling\"{u}ismo y de las relaciones entre 
	la lengua materna y la segunda lengua.
	\item Caracter\'{i}stica del castellano boliviano en sus diferentes niveles
	culturales
	
	\end{itemize}
\item Investigar sobre las lenguas, realizando estudios comparativos de sistemas
de comunicaci\'{o}n y estructuras de las lenguas en todos los niveles de 
ense\~{n}anza
\item Evaluador de contenidos de las asignaturas relacionadas con el \'{a}rea de
lenguas en todos los niveles de ense\~{n}anza
\item Desempe\~{n}ar eficientemente en cualquier otro campo en el que exija 
conocimiento y formaci\'{o}n de lenguas (Documento Carrera Ling\"{u}istica 
Aplicada a la Ense\~{n}anza de lenguas 2009)

\end{enumerate}

El perfil profesional del estudiante de la Carrera Ling\"{u}istica Aplicada a la
Ense\~{n}anza de lenguas, se\~{n}ala que los objetivos est\'{a}n enfocados en su
mayor\'{i}a en el \'{a}rea de t\'{e}cnicas que corroboran al \'{a}mbito educativo
a trav\'{e}s de la ense\~{n}anza de las lenguas(L1 y L2).

Este documento a\'{u}n se mantiene en desarrollo y sin modificaci\'{o}n alguna.
\cite{Q2014}

\section{Objetivos Generales}

El licenciado en LAEL ser\'{a} capaz de:

\begin{itemize}

\item Desenvolverse en su medio como agente de cambio, comprometi\'{e}ndose 
profesionalmente con este.
\item Analizar la realidad educativa nacional, particularmente la ling\"{u}istica,
proponiendo metodolog\'{i}as espec\'{i}ficas para la ense\~{n}anza de lenguas( 
castellano, lengua nativa y/o extranjera).
\item Planificar la ense\~{n}anza de las lenguas extranjeras, del castellano y 
del quechua u otra lengua ind\'{i}gena, en los diferentes niveles de ense\~{n}anza
del Sistema Educativo Plurinacional: inicial, primario, secundario y universitario.
\item Evaluar, dise\~{n}ar y/o adaptar materiales did\'{a}cticos para la 
ense\~{n}anza de lenguas.

\end{itemize}

\section{Mercado Profesional}

Los profesionales egresados de la Carrera LAEL podr\'{a}n desempe\~{n}arse 
profesionalmente en las siguientes \'{a}reas:

\subsection{Ense\~{n}anza de Lenguas}

\begin{description}

\item[Educaci\'{o}n Primaria y Secundaria:] Ense\~{n}anza de ingl\'{e}s y 
franc\'{e}s como lengua extranjera; ense\~{n}anza de castellano y quechua
como lengua materna o seguna lengua.
\item[Educaci\'{o}n superior:] Ense\~{n}anza de ingl\'{e}s y franc\'{e}s como 
lenguas extranjeras; ense\~{n}anza de quechua como segunda lengua y ense\~{n}anza
de castellano como lengua materna.

\end{description}

\subsection{Dise\~{n}o y planificaci\'{o}n}

\begin{itemize}

\item Dise\~{n}o curricular.
\item Dise\~{n}o de planes y programas de estudio de lenguas, materna, segunda
y extranjera.
\item Dise\~{n}o de materiales did\'{a}cticos.
\item Adaptaci\'{o}n y adecuaci\'{o}n de materiales did\'{a}cticos.

\end{itemize}

\subsection{Evaluaci\'{o}n}

\begin{itemize}

\item Evaluaci\'{o}n de procesos de ense\~{n}anza y aprendizaje.
\item Evaluaci\'{o}n de materiales de ense\~{n}anza de lenguas.
\item Evaluaci\'{o}n de programas de ense\~{n}anza de lenguas.

\end{itemize}