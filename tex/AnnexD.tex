\chapter{Licencia Pública General v.1} \label{chap:LPG-Bolivia}

\section{Introducción}

Esta licencia está basada en la Licencia Pública General GNU (GNU GPL) de la
Fundación para el Software Libre (www.fsf.org), la cual ha sida adaptada por
la agencia para el Desarrollo de la Sociedad de la Información en Bolivia 
(ADSIB) a la normativa legal vigente en Bolivia, enmarcada en la Ley General
de Telecomunicaciones, Tecnologías de la Información y Comunicación, Ley 164
de 8 de agosto de 2011 y el Reglamento por el Decreto Supremo 1793 de 13 de
Noviembre de 2013. \cite{LPGBolivia}
 
\section{Preámbulo}

Hablar de software libre, se refiere a la libertar de acción, no de precio.
La LPG-Bolivia esta diseñada para garantizar la libertad de distribuir
copias de software libre (cobrar por ello si quiere). Los desarrolladores
que usan la LPG-Bolivia protegen tus derechos con dos pasos. \cite{LPGBolivia}

\begin{itemize}

\item Haciendo valer el derecho de propiedad intelectual en el software.

\item Ofrece esta licencia que le da permiso legal para copiarlo,
distribuirlo y/o modificar-lo.
\end{itemize}

\section{Términos y Condiciones}

A continuación se referencia los beneficios de la licencia, para mayor
especificación revisar el documento: \cite{LPGBolivia}.

\subsection{Código Fuente}

El \textquotedouble{código fuente} de una obra es el formato preferido de la
misma para realizar modificaciones sobre ella. \textquotedouble{Código Objeto}
se refiere a cualquier formato de la obre que no sea código fuente.
\cite{LPGBolivia}

\subsection{Transmisión de Copias Legales}

Se podrá cobrar cualquier importe o no cobrar nada por cada copia que
transmita y se podrá ofrecer soporte o protección de garantías mediante un
pago. \cite{LPGBolivia}