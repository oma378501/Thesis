\chapter{Aprendizaje Autorregulado promovido por Carrera LAEL}

\section{Autor regulación y Aprendizaje Autorregulado}

Partiendo del interés de un aprendiz autorregulado como toda persona que desea
mejorar su conocimiento en una lengua, considerando el uso de un recurso como
el Podcast en su tiempo libre, este llega a realizar una planificación,
seguimiento y control de su evaluación en el transcurso del tiempo.

Para hablar de este tema, es importante remitirse a los trabajos de Schunk y
Zimmerman. Para Schunk la autor regulación consiste en los pensamiento,
sentimientos y actos originados por los estudiantes que están orientados
sistemáticamente a la consecución de metas. Ziemmerman, por su parte, la define
como el grado en el cual los individuos son participantes activos en su propio
proceso de aprendizaje desde el punto de vista meta cognitivo, motivacional y
realtivo a su comportamiento. \cite{rinconconsideraciones}

\section{Descripción de la unidad patrocinadora; Carrera LAEL}

La Carrera de LAEL, lleva 35 años de trayectoria al interior de la Facultad de
Humanidades y Ciencias de la Educación, tuvo un recorrido histórico-
trascendental, teniendo en cuenta las raíces que obtuvo para su surgimiento en
el año 1972, brindando materias de servicio dentro la Facultad de Ciencias
Puras y Naturales, en la UMSS. \cite{CMNPZ2014}

\subsection{Perfil Profesional del Estudiante en la Carrera LAEL}

El estudiante en la Carrera LAEL el documento denominado \textquotedouble{
Correspondencia y recibida y Despachada} del año 2009, destacó el Perfil
profesional del Estudiante de la Carrera LAEL constituido textualmente de la
siguiente manera:

\begin{enumerate}

\item Un profesional comprometido con su medio en el que gracias a procesos de
investigación de la realidad boliviana aplicara métodos y técnicas
adecuados dentro del proceso de enseñanza aprendizaje de lenguas en el 
sistema educativo nacional y universitario.
\item Ser capaz de evaluar y adaptar métodos de enseñanza, tanto para
las lenguas extranjeras como para el castellano y el Quechua: lengua extranjera
y/o segunda lengua.
\item Realizar investigación multidisciplinaria para estudios e 
interpretación sobre:
	\begin{itemize}
	
	\item La enseñanza de lenguas en el sistema educativo.
	\item Problemas de alfabetización en nuestro país, aportando desde la
	perspectiva de las lenguas.
	\item Problemas específicos de bilingüísmo y de las relaciones entre 
	la lengua materna y la segunda lengua.
	\item Característica del castellano boliviano en sus diferentes niveles
	culturales.
	
	\end{itemize}
\item Investigar sobre las lenguas, realizando estudios comparativos de sistemas
de comunicación y estructuras de las lenguas en todos los niveles de 
enseñanza.
\item Evaluador de contenidos de las asignaturas relacionadas con el área de
lenguas en todos los niveles de enseñanza.
\item Desempeñar eficientemente en cualquier otro campo en el que exija 
conocimiento y formación de lenguas (Documento Carrera Lingüística 
Aplicada a la Enseñanza de lenguas 2009).

\end{enumerate}

El perfil profesional del estudiante de la Carrera LAEL, señala que los
objetivos están enfocados en su mayoría en el área de técnicas que corroboran
al ámbito educativo a través de la enseñanza de las lenguas(L1 y L2).

Este documento aún se mantiene en desarrollo y sin modificación alguna.
\cite{Q2014}

\subsection{Objetivos Profesionales}

A continuación se da a conocer los objetivos.
  
EL Licenciado en LAEL será capaz de:

\begin{itemize}

\item Interpretar la actualidad de la educación nacional, particularmente en
la lingüística, proponiendo metodologías especificas para la enseñanza de la
lengua nativa y/o extranjera. 
\item Analizar e interpretar la realidad educativa, regional y particularmente
la lingüística.
\item Proponer metodologías especificas para la enseñanza de la lengua
extranjera, del Castellano y del Quechua.
\item Planificar la enseñanza de las lenguas en los diferentes niveles de
enseñanza del Sistema Educativo Nacional: inicial, primario, secundario y
universitario.
\item Evaluar, diseñar y/o adaptar material de apoyo para la enseñanza de
lenguas extranjeras.

\end{itemize}

A través de los objetivos profesionales se desarrolla y facilita, en los
estudiantes, la comprensión de la Carrera de (LAEL) para favorecer, y
desenvolverse sin dificultad en nuestro contexto. \cite{CMNPZ2014}