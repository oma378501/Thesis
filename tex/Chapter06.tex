\chapter{CONCLUSIONES Y RECOMENDACIONES}

Como consecuencia del Proyecto Adscripci\'{o}n, considerado como Trabajo 
interdiciplinario se tiene las siguientes consideraciones:

\begin{itemize}

\item Se sugiere cuando se quiera configurar un servicio de autentificaci\'{o}n
externo como oauth en redes sociales como ser: Facebook, Google, Twitter es
conveniente realizar pruebas de configuraci\'{o}n a equipos con IP p\'{u}blica.
\item Se considera importante brindar un servidor de producci\'{o}n para poder 
realizar funcionalidad de servicio correo tomando un servidor SMTP externo,
cron job (tareas segundo plano), caso contrario solicitar autorizaci\'{o}n 
escrita al responsable en la Unidad Patrocinadora (Ling\"{u}istica).
\item Si se desea realizar pruebas con extractores de microformatos como ser:
http://pin13.net/ es necesario que el equipo pueda tener una IP p\'{u}blica.
\item Si se desea agregar sem\'{a}ntica al contenido web que se tiene, se 
sugiere participar de chat online IRC irc://irc.freenode.net/microformats para
compartir sugerencias de terceros.
\item Si el sistema a implementar es Web se sugiere utilizar una distribuci\'{o}n
Linux como entorno desarrollo para luego realizar la transferencia de 
tecnolog\'{i}a a un servidor de producci\'{o}n.
\item Es conveniente realizar comparaciones de framework(s) tomando los siguientes
aspectos: cantidad de miembros en la comunidad, si se utiliza un DBMS relacional
tomar en cuenta que el mismo tenga soporte para generaci\'{o}n de llaves primarias
compuestas, si el framework cuenta con un m\'{o}dulo para la generaci\'{o}n de pruebas
unitarias, funcionales, integraci\'{o}n.
\item Se recomienda cuando se tiene un proyecto de adscripci\'{o}n utilizar
framework(s) como est\'{a}ndar de trabajo debido que el proyecto se realizo por
dos personas adicionales, con el objetivo de tener documentaci\'{o}n para la 
implementaci\'{o}n del proyecto basado en componentes.
\item Se ve por conveniente realizar pol\'{i}ticas internas dentro el equipo de
trabajo como implementaci\'{o}n en un solo lenguaje (ingl\'{e}s o espa\~{n}ol)
y tambi\'{e}n recurrir al manejo de versionadores de c\'{o}digo (git), 
manejadores de tareas gratuitos online (pivotal tracker).
\item Se recomienda tener una buena comunicaci\'{o}n con el Coordinador de la
Unidad Patrocinadora, realizar reuniones de presentaci\'{o}n
de Autoridades: Directores de Carrera, Tutor Proyecto, Adscritos. Para poder 
ver el estado del proyecto y dar sugerencias si fueran oportunas.
\item Tener un seguimiento respaldado por documentos entre las partes, Adscritos
y Coordinador para luego pedir la carta de conclusi\'{o}n del proyecto, Adem\'{a}s
de ser disciplinado y promover habilidades \'{a}giles en desarrollo de software con tu
equipo de trabajo, tomar responsabilidad de las consecuencias que lleva realizar un
proyecto adscripci\'{o}n debido que un estudiante va en representaci\'{o}n de su
Unidad Origen (Inform\'{a}tica).

\end{itemize}

\section{Trabajos Futuros}

Se recomenienda las siguientes experiencias para proyectos similares:

\begin{itemize}

\item Si se trata de un proyecto multimedia se debe utilizar servidores web 
especializados como ser: nginx \footnote{ngnix: Es conocido por su alto 
rendimiento, la estabilidad, la gran variedad de funciones, configuraci\'{o}n
simple, y bajo consumo de recursos} o tal vez lighttpd \footnote{lighttpd: 
Esta dise\~{n}ado y optimizado para entornos de alto rendimiento. Con una 
peque\~{n}a huella de memoria en comparaci\'{o}n con otros servidores web, la
gesti\'{o}n eficaz de la CPU de carga y avanzando lighttpd conjunto de 
caracter\'{i}sticas es la soluci\'{o}n perfecta para cada servidor que est\'{a}
sufriendo problemas de carga}
\item Si se desea crear material multimedia como imagenes, historietas, producci\'{o}n
de audio/video es aconsejable utilizar herramientas denoninadas openSource 
\footnote{openSource: Se refiere a software que est\'{a} desarrollado, probado,
o mejorar a trav\'{e}s de la colaboraci\'{o}n p\'{u}blica y distribuye con la
idea de que la deben compartir con los dem\'{a}s, lo que garantiza una 
colaboraci\'{o}n futuro abierto} debido que estos recursos tienen que ser usados
para utilizar difusi\'{o}n, tomando en cuentas las caracter\'{i}sticas propias 
de un software open source, como ser: multiplataforma, optimizaci\'{o}n de recursos
para poder realizar en un ancho de banda limitado o conexiones lentas. 
\end{itemize}