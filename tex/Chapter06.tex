\chapter{CONCLUSIONES Y RECOMENDACIONES}

Como consecuencia del Proyecto Adscripci\'{o}n, considerado como Trabajo 
interdiciplinario se tiene las siguientes conlusiones y sugerencias:

\begin{itemize}

\item Se provee de un servicio agregador de noticias por programa aprendizaje
para poder recibir notificaciones de nuevos contenidos v\'{i}a correo 
electr\'{o}nico, tambi\'{e}n se puede realizar la eliminaci\'{o}n de 
subscripci\'{o}n.   
\item Se provee sincronizaci\'{o}n de audio con transcripci\'{o}n origen y 
destino. Adem\'{a}s se provee una gesti\'{o}n de glosario por Podcast Audio. 
\item Se agrega contenido sem\'{a}ntico sobre transcripciones origen y destino
, adem\'{a}s de agregar sem\'{a}ntica en glosario.
\item Se implementa pruebas funcionales con el objetivo de brindar robustes dentro
la l\'{o}gica de subscripci\'{o}n.
\item Si se desea configurar un servicio de autentificaci\'{o}n externo como 
oauth en redes sociales como ser: Facebook, Google, Twitter es conveniente 
realizar pruebas de configuraci\'{o}n a equipos con IP p\'{u}blica.
\item Se considera importante disponer un servidor de producci\'{o}n para poder 
realizar funcionalidad de servicio correo tomando un servidor SMTP externo,
cron job (tareas segundo plano), caso contrario solicitar autorizaci\'{o}n 
escrita al responsable en la Unidad Patrocinadora (Ling\"{u}\'{i}stica).
\item Si se desea realizar pruebas con extractores de microformatos como ser:
http://pin13.net/ es necesario que el equipo deba tener IP p\'{u}blica.
\item Si se opta por agregar sem\'{a}ntica al contenido web que se tiene, se 
sugiere participar de chat online IRC irc://irc.freenode.net/microformats para
compartir sugerencias de terceros.
\item Si el sistema a implementar es Web se sugiere utilizar una distribuci\'{o}n
Linux como entorno desarrollo para luego realizar la transferencia de 
tecnolog\'{i}a a un servidor de producci\'{o}n.
\item Si se opta por el uso de framework(s) tomar los siguientes aspectos: 
cantidad de miembros en la comunidad, curva de aprendizaje. Si el DBMS es 
relacional verificar si cuenta con soporte para generaci\'{o}n de llaves 
primarias compuestas, adem\'{a}s de tener un m\'{o}dulo para la 
generaci\'{o}n de pruebas.
\item Si se opta por utilizar framework(s) como est\'{a}ndar de trabajo debido
que se cuenta con documentaci\'{o}n en la etapa de implementaci\'{o}n, la 
cantidad de personas es mayor a dos o m\'{a}s basar el trabajo en componentes
contemplando un tiempo de aprendizaje.
\item Se reconomienda realizar pol\'{i}ticas internas dentro el equipo de 
trabajo como convenciones de variables, nombre de fuciones y tambi\'{e}n 
recurrir al manejo de versionadores de c\'{o}digo (git), manejadores de tareas
gratuitos Online (pivotal tracker).
\item Se recomienda tener una buena comunicaci\'{o}n con el Coordinador de la
Unidad Patrocinadora, realizar reuniones de presentaci\'{o}n
de Autoridades: Directores de Carrera, Tutor Proyecto, Adscritos. Para poder 
ver el estado del proyecto y dar sugerencias si fueran oportunas.
\item Tener un seguimiento respaldado por documentos entre las partes, Adscritos
y Coordinador para luego pedir la carta de conclusi\'{o}n del proyecto, Adem\'{a}s
de ser disciplinado y promover habilidades \'{a}giles en desarrollo de software con tu
equipo de trabajo, tomar responsabilidad de las consecuencias que lleva realizar un
proyecto adscripci\'{o}n debido que un estudiante va en representaci\'{o}n de su
Unidad Origen (Inform\'{a}tica).
\item Se considera cuando se desee implementar un proyecto adscripci\'{o}n para 
una Unidad 
Patrocinadora de la UMSS, se debe optar por un docente respecto al \'{a}rea
del proyecto que se encuentre trabajando tiempo completo en la universidad debido
que se necesite solicitar reuniones, participaci\'{o}n como autoridad responsable
del proyecto, si el proyecto esta compuesto por m\'{a}s de dos personas, es 
aconsejable que el tutor sea de ambos y tenga un solo criterio al momento 
definir los obejetivos a desarrollar.

\end{itemize}

\section{Trabajos Futuros}

Se recomenienda las siguientes experiencias para proyectos similares:

\begin{itemize}

\item Si se trata de un proyecto multimedia se debe utilizar servidores web 
especializados como ser: nginx \footnote{ngnix: Es conocido por su alto 
rendimiento, la estabilidad, la gran variedad de funciones, configuraci\'{o}n
simple, y bajo consumo de recursos \cite{nginx}}  o tal vez lighttpd \footnote{lighttpd: 
Esta dise\~{n}ado y optimizado para entornos de alto rendimiento. Con una 
peque\~{n}a huella de memoria en comparaci\'{o}n con otros servidores web, la
gesti\'{o}n eficaz de la CPU de carga y avanzando lighttpd conjunto de 
caracter\'{i}sticas es la soluci\'{o}n perfecta para cada servidor que est\'{a}
sufriendo problemas de carga \cite{lighttps}} 
\item Si se desea crear material multimedia como imagenes, historietas, producci\'{o}n
de audio/video es aconsejable utilizar herramientas denoninadas openSource 
\footnote{openSource: Es software cuyo c\'{o}digo fuente est\'{a} disponible 
para su modificaci\'{o}n o mejora por parte de nadie \cite{openSource}}  debido 
que estos recursos tienen que ser usados para utilizar difusi\'{o}n, tomando 
en cuentas las caracter\'{i}sticas propias de un software open source, como 
ser: multiplataforma, optimizaci\'{o}n de recursos para poder realizar en un 
ancho de banda limitado o conexiones lentas. 
\end{itemize}