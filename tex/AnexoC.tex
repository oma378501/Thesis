\chapter{Documento de validaci\'{o}n y aceptaci\'{o}n 26-01-2015}

El presente documento tiene por finalidad de constatar la conformidad tanto del
Cliente como el equipo de desarrollo.

Esta es la presentaci\'{o}n de la versi\'{o}n Alpha estipulado en la propuesta 
t\'{e}cnica presentada Correspondiente al Sprint segundo, tercero \textquotedblleft
Versi\'{o}n Alpha listo\textquotedblright.

Dicha presentaci\'{o}n est\'{a} sujeto al pago del 4 (porciento) del total del
costo del sistema, para Mayor conformidad para ambas partes se remarca los 
siguientes aspectos

\begin{enumerate}

\item \textbf{Escala de evaluaci\'{o}n}

\begin{table}[!h]
\centering
\begin{tabular}{|c|c|c|}
\hline
\textbf{Puntaje} & \textbf{Valor Num\'{e}rico} & \textbf{Concepto} \\ \hline
NA & 0 & \begin{tabular}[c]{@{}c@{}}Se necesita hacer cambios\\ profundos\end{tabular} \\ \hline
RA & 1 & \begin{tabular}[c]{@{}c@{}}Requiere modificaci\'{o}n para\\ aceptar\end{tabular} \\ \hline
A & 2 & Aceptado \\ \hline
\end{tabular}
\end{table}

\item \textbf{Las funcionalidades a presentar son las siguientes:}

\begin{itemize}

	\item Reiniciar Contrase\~{n}a de Usuario (Usuario habilitado).
		\begin{itemize}
			\item Envi\'{o} de solicitud de cambio de contrase\~{n}a.
		\end{itemize}
	\item Gesti\'{o}n rol usuario Coordinador (Usuario habilitado)
	\item Gestionar Categor\'{i}a (Usuario autentificado)
	\item Gesti\'{o}n rol usuario Tutor(Usuario habilitado, Sub categor\'{i}a creada).
	\item Gestionar Contenido por intereses(Tener rol Tutor, Estar asignado a 
	subcategor\'{i}a).
	\item Generar men\'{u} de tipos de contenido por Categor\'{i}a(Contenido publicado)
		
\end{itemize}

\item \textbf{El pago del 4 (porciento) se dar\'{a} por hecho si se est\'{a} conforme 
con la mitad m\'{a}s uno de los puntos}

\end{enumerate}

Si est\'{a} conforme con una funcionalidad si se cumple los siguientes aspectos
estipulados en el sumario de evaluaci\'{o}n de p\'{a}rrafos.

\section{Sumario de Evaluaci\'{o}n de par\'{a}metro}

\subsection{Reiniciar Contrase\~{n}a de Usuario}

\textbf{Envi\'{o} de solicitud de cambio de contrase\~{n}a}

Requisito: Ser usuario habilitado, caso contrario utilizar las credenciales del
usuario por defecto; usuario: Juan Omar Huanca Balboa contrase\~{n}a: autorregulado.

Abra un navegador web con la siguiente direcci\'{o}n (URL): plataforma 
\footnote{plataforma: http:// plataformaeducativalael.hum.umss.edu.bo} Pinche en
el enlace Iniciar Sesi\'{o}n ubicado en la parte superior derecha, seguido de
pinchar sobre en enlace Recuperar Contrase\~{n}a.

\begin{minipage}[b]{\hsize}\centering
\begin{tabular}{|l|l|l|}
\hline
\multicolumn{1}{|c|}{\textbf{N}} & \multicolumn{1}{c|}{\textbf{Parametros a evaluar}} & \multicolumn{1}{c|}{\textbf{Puntaje}} \\ \hline
01 & \begin{tabular}[c]{@{}l@{}}Si Ud. eligi\'{o} la cuenta por defecto llene, correo:\\ omar.huanca.balboa@gmail.com, contrase\~{n}a:\\ omar.huanca.balboa. Captcha, pinchar sobre el bot\'{o}n Enviar\\ entonces ejecutar paso 03; caso contrario ingrese el correo\\ electr\'{o}nico relacionado con la cuenta habilitada.\end{tabular} &  \\ \hline
02 & \begin{tabular}[c]{@{}l@{}}\textquestiondown Ud. puede ingresar datos en los campos: Direcci\'{o}n de Correo,\\ Captcha pinchar sobre el bot\'{o}n Enviar ?\end{tabular} &  \\ \hline
03 & \begin{tabular}[c]{@{}l@{}}\textquestiondown Ud. puede leer un mensaje \textquotedblleft Un correo electr\'{o}nico que\\ contiene m\'{a}s instrucciones ha sido enviada bandeja de direcci\'{o}n\\ de correo electr\'{o}nico asociada con su cuenta de proveedor\textquotedblright ?\end{tabular} &  \\ \hline
04 & \begin{tabular}[c]{@{}l@{}}\textquestiondown Ud. ingrese a su bandeja de mensajes del correo electr\'{o}nico\\ ingresado anteriormente, un mensaje de Administrador -\\ Recuperar Contrase\~{n}a en el cuerpo un enlace?\end{tabular} &  \\ \hline
05 & \begin{tabular}[c]{@{}l@{}}\textquestiondown Ud. pinche sobre el enlace le enviara a un formulario con los\\ siguientes campos: Contrase\~{n}a, Repetir Contrase\~{n}a?\end{tabular} &  \\ \hline
06 & \begin{tabular}[c]{@{}l@{}}\textquestiondown Ud. ingrese la nueva contrase\~{n}a dos veces en los campos:\\ Contrase\~{n}a, Repetir Contrase\~{n}a y pinchar sobre el bot\'{o}n Enviar,\\ le muestra el siguiente mensaje \textquotedblleft Su direcci\'{o}n de correo ha sido\\ verificado\textquotedblright ?\end{tabular} &  \\ \hline
07 & \begin{tabular}[c]{@{}l@{}}\textquestiondown Si Ud. eligi\'{o} la cuenta de usuario por defecto (autorregulado)\\ entonces ingrese la nueva contrase\~{n}a pinche sobre el bot\'{o}n\\ Iniciar Sesi\'{o}n con sus nuevas credenciales caso contrario\\ ingrese la cuenta de usuario de la cuenta habilitada?\end{tabular} &  \\ \hline
\end{tabular}
\end{minipage}

Nota: Si est\'{a} conforme con la funcionalidad \textquotedblleft Recobrar Contrase\~{n}a\textquotedblright si la sumatoria
de Puntaje tiene un valor mayor y/o igual 4

Las siguiente funcionalidad tiene que solicitar webmaster (administrador) que
agregue el rol coordinador en una cuenta habilitada, caso contrario puede hacer
uso de la cuenta por defecto nombre usuario: coordinador, contrase\~{n}a: 
coordinador.

Ingrese la siguiente direcci\'{o}n URL en su navegador: plataforma 
\footnote{plataforma: http:// plataformaeducativalael.hum.umss.edu.bo}

Pinche en el enlace Iniciar Sesi\'{o}n ubicado en la parte superior derecha

\section{Gesti\'{o}n rol usuario Coordinador}

\begin{minipage}[b]{\hsize}\centering
\begin{tabular}{|l|l|l|}
\hline
\multicolumn{1}{|c|}{\textbf{N}} & \multicolumn{1}{c|}{\textbf{Parametros a evaluar}} & \multicolumn{1}{c|}{\textbf{Puntaje}} \\ \hline
01 & \begin{tabular}[c]{@{}l@{}}\textquestiondown Si Ud. Eligio la cuenta habilitada tiene que dar a conocer al\\ webmaster a cual sub categor\'{i}a quiere estar asignado (Quechua \\ B\'{a}sico, Quechua Psicosocial, Frances B\'{a}sico, English Spread in \\ Cochabamba)? \\ \end{tabular} &  \\ \hline
02 & \begin{tabular}[c]{@{}l@{}}\textquestiondown Ud. Inicie sesi\'{o}n con las credenciales de la cuenta habilitada,\\ 
deber\'{i}a de aparecer nuevas opciones en el men\'{u} derecho de su \\ p\'{a}gina principal de sesi\'{o}n?\end{tabular} &  \\ \hline
03 & \begin{tabular}[c]{@{}l@{}}\textquestiondown Ud. Puede solicitar al webmaster verbalmente que se pueden\\ revocar los privilegios asignados a la cuenta habilitada\\ mencionada anteriormente?\end{tabular} &  \\ \hline
\end{tabular}
\end{minipage}

Nota: Si est\'{a} conforme con la funcionalidad \textquotedblleft Gesti\'{o}n 
rol usuario Coordinador\textquotedblright si la sumatoria de Puntaje tiene un
valor mayor y/o igual 2

La siguiente funcionalidad tiene ingresar con una cuenta habilitada que tenga 
el rol de coordinador asignado correspondientemente. Caso contrario ingrese con
las credenciales Nombre Usuario: coordinador, Contrase\~{n}a: coordinador.

Ingrese la siguiente direcci\'{o}n URL en su navegador: plataforma 
\footnote{plataforma: http:// plataformaeducativalael.hum.umss.edu.bo}
o pinche sobre la opci\'{o}n Inicio ubicado en el men\'{u} principal de la 
p\'{a}gina de presentaci\'{o}n Principal.

Pinche en el enlace Iniciar Sesi\'{o}n ubicado en la parte superior derecha.

\section{Gestionar Categor\'{i}a}

\begin{minipage}[b]{\hsize}\centering
\begin{tabular}{|l|l|l|}
\hline
\multicolumn{1}{|c|}{\textbf{N}} & \multicolumn{1}{c|}{\textbf{Parametros a evaluar}} & \multicolumn{1}{c|}{\textbf{Puntaje}} \\ \hline
01 & \begin{tabular}[c]{@{}l@{}}\textquestiondown Ud. Ingrese sus credenciales en los campos: Nombre Usuario,\\ Contrase\~{n}a, le aparecer\'{a} una opci\'{o}n de su cuenta\\ administrativa?\end{tabular} &  \\ \hline
02 & \begin{tabular}[c]{@{}l@{}}\textquestiondown Ud. Pinche sobre la opci\'{o}n Categoria en la sub opci\'{o}n\\ Registrar Categor\'{i}a y le muestra un formulario?\end{tabular} &  \\ \hline
03 & \begin{tabular}[c]{@{}l@{}}\textquestiondown Ud. Ingrese el nombre de la categor\'{i}a padre con la categor\'{i}a\\ 
reflexiva opci\'{o}n Padre, Archivo Respuesta Correcta, Archivo\\ Respuesta Incorrecta, pinchar sobre el bot\'{o}n Registar, a\\
continuaci\'{o}n deber\'{i}a mostrarle un detalle de la categor\'{i}a padre\\ registrada?\end{tabular} &  \\ \hline
04 & \begin{tabular}[c]{@{}l@{}}\textquestiondown Ud. Pinche sobre la opci\'{o}n Categor\'{i}a en la sub opci\'{o}n\\ Registrar Categoria y le muestra un formulario?\end{tabular} &  \\ \hline
05 & \begin{tabular}[c]{@{}l@{}}\textquestiondown Ud. Ingrese el nombre de la categor\'{i}a hija con la categor\'{i}a\\ reflexiva opci\'{o}n Hijo, Nivel Categor\'{i}a, Imagen Categor\'{i}a,\\ Descripci\'{o}n Categor\'{i}a, Descripci\'{o}n Cr\'{e}ditos, Descripci\'{o}n\\ Objetivos, pinchar sobre el bot\'{o}n Registar, a continuaci\'{o}n\\ deberia mostrarle un detalle de la categor\'{i}a hijo registrado?\end{tabular} &  \\ \hline
06 & \begin{tabular}[c]{@{}l@{}}\textquestiondown Ud. Pinche sobre la opci\'{o}n Categor\'{i}a en la sub opci\'{o}n\\ Administrar Categor\'{i}a y le muestra una tabla de categorias\\ registradas?\end{tabular} &  \\ \hline
07 & \begin{tabular}[c]{@{}l@{}}\textquestiondown Ud. elija una opci\'{o}n de la tabla, seguido podra aplicar las\\ siguiente funcionalidades: Ver, Actualizar, Eliminar. Como\\ primera opci\'{o}n pinche sobre la opci\'{o}n Ver ubicado en la parte\\ derecha bajo la etiqueta Acci\'{o}n, podra observar un detalle de los\\ campos con que cuenta la categor\'{i}a?\end{tabular} &  \\ \hline
08 & \begin{tabular}[c]{@{}l@{}}\textquestiondown Ud. Pinche sobre la opci\'{o}n Categor\'{i}a en la sub opci\'{o}n\\ Administrar Categor\'{i}a y le muestra una tabla de categorias\\ registradas? \end{tabular} & \\ \hline
09 & \begin{tabular}[c]{@{}l@{}}\textquestiondown Ud. elija una opci\'{o}n de la tabla, como segunda opci\'{o}n pinche\\ sobre la acci\'{o}n Actualizar ubicado en la parte derecha bajo la\\ etiqueta Acci\'{o}n, podra editar los campos previamente\\ registrados?\end{tabular} & \\ \hline
10 & \begin{tabular}[c]{@{}l@{}}\textquestiondown Ud. Pinche sobre la opci\'{o}n Categor\'{i}a en la sub opci\'{o}n\\ Administrar Categor\'{i}a y le muestra una tabla de categorias\\ registradas?\end{tabular} & \\ \hline
11 & \begin{tabular}[c]{@{}l@{}}\textquestiondown Ud elija una opci\'{o}n de la tabla, como tercera opci\'{o}n pinche\\ sobre la opci\'{o}n acci\'{o}n Borrar ubicado en la parte derecha bajo\\ la etiqueta Acci\'{o}n, un mensaje de confirmaci\'{o}n le saldra\\ pidiendole aceptar o denegar?\end{tabular} & \\ \hline
12 & \begin{tabular}[c]{@{}l@{}}\textquestiondown Ud. Pinche sobre la opci\'{o}n Categor\'{i}a en la sub opci\'{o}n\\ Administrar Categor\'{i}a y le muestra una tabla de categorias\\ registradas?\end{tabular} & \\ \hline
13 & \begin{tabular}[c]{@{}l@{}}\textquestiondown Ud. Pinche sobre la opci\'{o}n Categor\'{i}a en la sub opci\'{o}n Listar\\ Categor\'{i}a y le muestra una lista de categorias registradas?\end{tabular} & \\ \hline
14 & \begin{tabular}[c]{@{}l@{}}\textquestiondown Ud. Podra ordenar las categorias seg\'{u}n el siguiente criterio:\\ Nombre Categor\'{i}a, ubicado en la parte superior, inferior de la\\ lista?\end{tabular} & \\ \hline
\end{tabular}
\end{minipage}

Nota: Si est\'{a} conforme con la funcionalidad \textquotedblleft Gestionar 
Categor\'{i}a \textquotedblright si la sumatoria de Puntaje tiene un valor 
mayor y/o igual 7

Las siguiente funcionalidad tiene que solicitar webmaster (administrador) que 
agregue el rol tutor en una cuenta habilitada, caso contrario puede hacer uso
de la cuenta por defecto nombre usuario: tutor, contrase\~{n}a: tutor.

Ingrese la siguiente direcci\'{o}n URL en su navegador: pataforma 
\footnote{plataforma: http:// plataformaeducativalael.hum.umss.edu.bo} o pinche 
sobre la opci\'{o}n Inicio ubicado en el menu principal de la p\'{a}gina 
de presentaci\'{o}n Principal.

Pinche en el enlace Iniciar Sesi\'{o}n ubicado en la parte superior derecha.

\section{Gestion rol usuario Tutor}

\begin{minipage}[b]{\hsize}\centering
\begin{tabular}{|l|l|l|}
\hline
\multicolumn{1}{|c|}{\textbf{N}} & \multicolumn{1}{c|}{\textbf{Parametros a evaluar}} & \multicolumn{1}{c|}{\textbf{Puntaje}} \\ \hline
01 & \begin{tabular}[c]{@{}l@{}}\textquestiondown Si Ud. Eligio la cuenta habilitada tiene que dar a conocer al\\ webmaster a cual sub categor\'{i}a quiere estar asignado para poder\\  gestionar sus contenidos?\end{tabular} &  \\ \hline
02 & \begin{tabular}[c]{@{}l@{}}\textquestiondown Ud. Inicie sesi\'{o}n con las credenciales de la cuenta habilitada,\\ 
deber\'{i}a de aparecer las opciones en el men\'{u} derecho de su\\ p\'{a}gina principal de sesi\'{o}n con los nombres: Mis Contenidos,\\ Actividad?\end{tabular} &  \\ \hline
03 & \begin{tabular}[c]{@{}l@{}}\textquestiondown Ud. Puede solicitar al webmaster verbalmente que se pueden\\ revocar los privilegios asignados a la cuenta habilitada\\ mencionada anteriormente?\end{tabular} &  \\ \hline
\end{tabular}
\end{minipage}

Nota: Si est\'{a} conforme con la funcionalidad \textquotedblleft Gesti\'{o}n
rol usuario Tutor \textquotedblright si la sumatoria de Puntaje tiene un valor
mayor y/o igual 2

Para realizar la siguiente funcionalidad, tiene que cumplir con la anterior
funcionalidad (Gesti\'{o}n rol usuario Tutor)

Ingrese la siguiente direcci\'{o}n URL en su navegador: plataforma 
\footnote{plataforma: http:// plataformaeducativalael.hum.umss.edu.bo} o pinche
sobre la opci\'{o}n Inicio ubicado en el men\'{u} principal de la p\'{a}gina de 
presentaci\'{o}n Principal.

Pinche en el enlace Iniciar Sesi\'{o}n ubicado en la parte superior derecha.

\section{Gestionar Contenido por intereses}

\begin{minipage}[b]{\hsize}\centering
\begin{tabular}{|l|l|l|}
\hline
\multicolumn{1}{|c|}{\textbf{N}} & \multicolumn{1}{c|}{\textbf{Parametros a evaluar}} & \multicolumn{1}{c|}{\textbf{Puntaje}} \\ \hline
01 & \begin{tabular}[c]{@{}l@{}}\textquestiondown Ud. Ingrese sus credenciales en los campos: Nombre Usuario,\\ Contrase\~{n}a, le aparecer\'{a} una opci\'{o}n de su cuenta\\ administrativa?\end{tabular} &  \\ \hline
02 & \begin{tabular}[c]{@{}l@{}}\textquestiondown Ud. Pinche sobre la opci\'{o}n Mis Contenidos a en la sub opci\'{o}n\\ Registrar Mi Contenido y le muestra un formulario?\end{tabular} &  \\ \hline
03 & \begin{tabular}[c]{@{}l@{}}\textquestiondown Ud. Ingrese el T\'{i}tulo, Archivo Imagen, Seleccione el tipo de\\ contenido, Archivo Reproductor, Seleccione la sub categor\'{i}a,\\ Fecha de liberaci\'{o}n, Resumen, Archivo Resoluci\'{o}n(*), Archivo\\ Glosario, Archivo Diccionario, Cr\'{e}ditos pinchar sobre el bot\'{o}n\\ Registar,a continuaci\'{o}n deberia mostrarle un detalle de la\\ categor\'{i}a padre registrada?\end{tabular} &  \\ \hline
04 & \begin{tabular}[c]{@{}l@{}}\textquestiondown Ud. Pinche sobre la opci\'{o}n Mis Contenidos en la sub opci\'{o}n\\ Administrar Mis Contenidos y le muestra una tabla de\\ contenidos registradas?\end{tabular} &  \\ \hline
05 & \begin{tabular}[c]{@{}l@{}}\textquestiondown Ud. elija una opci\'{o}n de la tabla, como segunda opci\'{o}n pinche\\ sobre la acci\'{o}n Actualizar ubicado en la parte derecha bajo la\\ etiqueta Acci\'{o}n, podra editar los campos previamente\\ registrados?\end{tabular} &  \\ \hline
06 & \begin{tabular}[c]{@{}l@{}}\textquestiondown Ud. Pinche sobre la opci\'{o}n Mis Contenidos en la sub opci\'{o}n\\ Administrar Mis Contenidos y le muestra una tabla de\\ contenidos registradas?\end{tabular} &  \\ \hline
07 & \begin{tabular}[c]{@{}l@{}}\textquestiondown Ud elija una opci\'{o}n de la tabla, como tercera opci\'{o}n pinche\\ sobre la opci\'{o}n acci\'{o}n Borrar ubicado en la parte derecha bajo\\ la etiqueta Acci\'{o}n, un mensaje de confirmaci\'{o}n le saldra\\ pidiendole aceptar o denegar?\\\end{tabular} &  \\ \hline
08 & \begin{tabular}[c]{@{}l@{}}\textquestiondown Ud. Pinche sobre la opci\'{o}n Mis Contenidos en la sub opci\'{o}n\\ Listar Mis Contenidos y le muestra una lista de todos los\\ contenidos registrados? \end{tabular} & \\ \hline
09 & \begin{tabular}[c]{@{}l@{}}\textquestiondown Ud. Podra ordenar las categorias seg\'{u}n el siguiente criterio:\\ T\'{i}tulo, Fecha creaci\'{o}n, Fecha liberaci\'{o}n?\end{tabular} & \\ \hline
\end{tabular}
\end{minipage}

(*) Si usted define como tipo de contenido audio este campo es requerido.

Nota: Si est\'{a} conforme con la funcionalidad \textquotedblleft Gestionar 
Contenido por intereses\textquotedblright si la sumatoria de Puntaje tiene un 
valor mayor y/o igual 5

Ingrese la siguiente direcci\'{o}n URL en su navegador: plataforma 
\footnote{plataforma: http:// plataformaeducativalael.hum.umss.edu.bo} o pinche
sobre la opci\'{o}n Inicio ubicado en el men\'{u} principal de la p\'{a}gina de 
presentaci\'{o}n Principal o pinche sobre la opci\'{o}n Vista P\'{a}gina Principal
situado en la parte superior de su panel administrativo de su cuenta.

\section{Generar men\'{u} de tipos de contenido por Categor\'{i}a}

\begin{minipage}[b]{\hsize}\centering
\begin{tabular}{|l|l|l|}
\hline
\multicolumn{1}{|c|}{\textbf{N}} & \multicolumn{1}{c|}{\textbf{Parametros a evaluar}} & \multicolumn{1}{c|}{\textbf{Puntaje}} \\ \hline
01 & \begin{tabular}[c]{@{}l@{}}\textquestiondown Ud. Pinche sobre el tipo de categor\'{i}a Audio/Video si fuera el\\ caso con anterioridad, saldra un elemento respecto la categor\'{i}a\\  padre, pinche sobre el mismo y podra ver una lista de sub\\ categorias asignadas al correspondiente idioma?\end{tabular} &  \\ \hline
02 & \begin{tabular}[c]{@{}l@{}}\textquestiondown Ud. Pinche sobre la imagen u/o nombre de la sub categor\'{i}a ,\\ 
para poder ver contenido(s) creados relacionados con sub\\ categor\'{i}a?\end{tabular} &  \\ \hline
03 & \begin{tabular}[c]{@{}l@{}}\textquestiondown Ud. Podra ver una animaci\'{o}n de desplazamiento de izquieda a\\ derecha de los contenidos comprendidos en la sub categor\'{i}a, en\\ la parte inferior podra ver los cr\'{e}ditos y objetivos de todos los\\ episodios?\end{tabular} &  \\ \hline
\end{tabular}
\end{minipage}

Nota: Si est\'{a} conforme con la funcionalidad \textquotedblleft Generar 
men\'{u} de tipos de contenido por Categor\'{i}a\textquotedblright si la 
sumatoria de Puntaje tiene un valor mayor y/o igual 2

Si se cumple m\'{a}s de los 4 puntos de 6 que fueron presentados a 
continuaci\'{o}n procedemos a firmar este documento los representantes para su
aceptaci\'{o}n.

\begin{minipage}[b]{\hsize}\centering
\begin{tabular}{|l|l|}
\hline
\begin{tabular}[c]{@{}l@{}}..........................................\\ Lic. Manuel Camacho Arce\\ PRODUCT OWNER\end{tabular} & \begin{tabular}[c]{@{}l@{}}.........................................\\ Juan Omar Huanca Balboa\\ SCRUMMASTER\end{tabular} \\ \hline
\end{tabular}
\end{minipage}
