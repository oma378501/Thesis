\chapter{Introducción}

La ley general de derechos y políticas lingüísticas 269 \textquotedouble{promueve
la implementación de una lengua originaria según la región}. Esta
disposición brinda el marco legal que promueve la enseñanza, su difusión,
aunque su alcance inicialmente está restringido a funcionarios públicos y/o
privados. Entonces su implementación exige la elaboración de recursos
educativos adecuados para la enseñanza de lenguas originarias. Esta demanda
crece más si se añade la exigencia de aprender una lengua extranjera, como
el inglés o francés, para fines de promover el turismo, el comercio y la
formación profesional.

De acuerdo con esta conjetura expuesta, este proyecto de adscripción se
propone implementar la gestión de podcast de tipo audio/vídeo elaborado por
la Carrera de Lingüística Aplicada a la Enseñanza de Lenguas (LAEL) de la
Universidad Mayor de San Simón, mediante la creación de un mecanismo de
notificación, a través de correo electrónico, del nuevo material pedagógico.

Así que, la plataforma web educativa fue diseñado para la gestión de podcast.
la plataforma utiliza la licencia LPG-Bolivia propuesta por
la ADSIB \footnote{ADSIB: Agencia para el desarrollo de la sociedad de la
información en Bolivia. \cite{LPGBolivia}}. De acuerdo con \cite{LPGBolivia}
\textquotedouble{protege los derechos intelectuales y materiales de
software en instituciones públicas}.

\section{Antecedentes}

En el año 2013, la carrera de LAEL impulsó la elaboración de recursos digitales
educativos para el fortalecimiento de lenguas originarias, destinado a
funcionarios públicos y/o privados de la zona urbana del municipio de Cochabamba.

En razón a lo expuesto, el año 2014, la Carrera de LAEL firmó un proyecto de
adscripción con la Carrera de Informática. Se acordó que se conformarán equipos
de trabajo entre ambas carreras para la elaboración, difusión de audio y vídeo.
Los estudiantes de lingüística se encargaron de escoger los contenidos y
elaborar recursos pedagógicos de los podcasts; mientras los de informática
aportaría en su difusión a través de la red de Internet. Desde esa fecha, se
conformaron seis equipos de trabajo.

\section{Definición del problema}

La Carrera de LAEL llevan varios años produciendo materiales educativos para la
enseñanza de lenguas. Sin embargo, estos permanecen inaccesibles, para la
mayoría de la población a la que se supone que está dirigida. 

El material producido se queda almacenado en los estantes de la biblioteca
de LAEL, al cual los posibles beneficiarios no pueden acceder debido a las
condiciones de préstamo que impone la universidad y la escasa difusión de
ellos. 

En consecuencia la población externa e interesada en aprender una lengua
extranjera suele recurrir a material disponible vía Internet que en su
mayoría se realizó en el exterior. Por una parte, estos aunque beneficioso,
no siempre están adecuados a las condiciones y necesidades de los usuarios
en un contexto urbano/rural.

Por otra parte, los recursos educativos para el fortalecimiento de quechua
son escasos, difíciles de encontrar. El material creado por los estudiantes
si estuviera disponible, serían de gran provecho y ayuda a la población
demandante.
 				
Se puede definir el problema de investigación, por lo previamente mencionado
de la siguiente manera: La escasa difusión de los \textbf{recursos digitales
educativos} producido por la Carrera de LAEL dificulta el desarrollo del 
\textbf{aprendizaje autorregulado de las lenguas de francés, inglés y quechua}.

\section{Objetivo}

\subsection{General}

Implementar un \textbf{suscriptor de podcast} para el fortalecimiento del 
\textbf{aprendizaje autorregulado las lenguas francés, inglés y quechua} mediante
el desarrollo de una plataforma web educativa.

\subsection{Específicos}

\begin{itemize}

\item Proveer personalización para suscripción por programa de aprendizaje (categoría).
\item Implementar glosario y subtitulado de podcast.
\item Representar el uso de web semántica para glosario y subtitulado.
\item Realizar un prueba de unidad para suscripción, reproducción de audio y
reproducción de vídeo.

\end{itemize}

\section{Justificación}

\subsection{Tecnológica}

Los entornos virtuales de aprendizaje apoya el fortalecimiento y enseñanza de
una lengua, la plataforma web educativa realiza la difusión de los recursos
digitales educativos. 

\subsection{Social}

Los recursos digitales educativos creados en la Carrera de LAEL, al contrario
de los que se encuentran en Internet, están adecuados a la realidad de los
posibles beneficiarios del departamento, estos responden a sus necesidades y
se encuentran contextualizados al entorno cultural, social y lingüístico.

\subsection{Económica}

Los recursos digitales educativos son simples y gratuitos.

\section{Limitaciones}

\begin{itemize}

\item La plataforma no brinda un módulo avanzado de usuario, un rol de sistema
tiene acceso a los diferentes privilegios en comparación de otro.

\item La plataforma solamente gestiona: podcast, actividad, transcripción,
infografía; la plataforma no brinda creación de ningún tipo de recurso.

\item El suscripción de noticia utiliza la versión RSS 2.0 para la generación
de un canal de noticia se crea por categoría.

\item La plataforma es de tipo web no debe funcionar como sistema
de escritorio.

\item Realizar la baja la suscripción de un usuario será únicamente dentro
del sistema.

\item La conexión a Internet necesita de un servicio de un proveedor ISP
\footnote{ISP: Es una empresa que proporciona a los individuos y otras empresas
el acceso de Internet y Otros servicios relacionados, tales como la creación
de sitios web y de alojamiento virtual. \cite{isp}}. 

\end{itemize}