\chapter{INTRODUCCION}

\section{Introducción}

La Carrera de Ling\"{u}ística Aplicada a la Enseñanza de Lenguas (LAEL) de la Universidad Mayor de San Simon (UMSS)
forma recursos  personal acorde a su medio, proponiendo mecanismos para la ense\"{n}anza y aprendizaje de lenguas.
Profesionales comprometidos con el cambio e interpretar la realidad educativa desde una perspectiva linguistica  proveendo
metodolog\'{i}as especificas para la ense\"{n}anza de la lengua nativa y extranjera.A mediados de la gestión 2014 se elaboró
un material educativo enfocado en el desarrollo de las habilidades comunicativas como ser: hablar, escribir, leer y escuchar. 
Haciendo hincapié en el auditivo por medio de recursos multimedia (Podcast) educativos a raíz de un análisis de necesidades a 
funcionarios públicos y/o privados de la urbe de Cochabamba. Los estudiantes elaboraron recursos multimedia educativos enfocados 
en el aprendizaje autorregulado de la lengua Quechua.

Por tales motivos se propone proveer soporte tecnológico utilizando la difusión de canales de noticias de Podcast sujetos a una 
subscripci\'{o}n realizado por un Programa de Aprendizaje.Con apoyo de las Tecnologías de la Información y Comunicación (TICs) 
enfocado en la enseñanza se pretende apoyar el proceso de Aprendizaje Autorregulado de una lengua nativa o extranjera.

\section{Antecedentes}

Se tiene una diversificaci\'{o}n de alcances con la TICs para fortalecer a la tecnolog\'{i}a como un medio entre
el Profesor y Estudiante.

\textquestiondown C\'{o}mo pueden contribuir las TICs al desarrollo de propuestas pedag\'{o}gicas pertinentes?
\begin{itemize}
\item En contextos altamente dicersos y desigualdades no pueden haber una \'{u}nica respuesta,
si no m\'{u}ltiples para responder a las necesidades educativas de todos los estudiantes.
\item Considerar las respuestas diversas tanto en los "contenidos" de las TICs como en sus soportes o dispositivos (por ejemplo computadoras adaptadas para personas con discapacidad).
\item Las TICs pueden ser una herramienta \'{u}til para diversificar la ense\'{n}anza y el aprendizaje.\cite{severin2013enfoques}
\end{itemize}

Modelo representativo de educaci\'{o}n superior basado en el enfoque del propio estudiante

No obstante, si tal como hemos hecho antes, radicaliz\'{a}ramos la definici\'{o}n y nos fu\'{e}ramos a un extremo para poner
ejemplos, nos dar\'{i}amos cuenta de que los modelos actualmente m\'{a}s centrados en el estudiante son los que se basan, 
fundamentalmente, en el autoaprendizaje o la autoformaci\'{o}n.\cite{duart2000aprender}

Se describe trabajos interdisciplinarios similares como: 

La Carrera LAEL impuls\'{o} en la creaci\'{o}n de material multimedia bajo un estudio de necesidad a funcionarios 
públicos y/o privados para entablar comunicaci\'{o}n con personas quechua hablantes que migraron para contar con
beneficios como ser: hospital, colegio, juzgado, mercado de abasto, servicio de identificación personal, 
contextos dentro el área urbana de la ciudad de Cochabamba como producto salió una tesis en el año 2014.

En conclusi\'{o}n, el proyecto de la elaboraci\'{o}n y producci'{o}n de Podcast va direccionado a cubrir
las necesidades de comunicaci\'{o}n en la lengua Quechua particularmente de los funcionarios tanto p\'{u}plicos
como privados de la zona urbana de la ciudad de Cochabamba. Se espera que los resultados de este proyecto contribuya
en cierta medida a la revitalizaci\'{o}n de la lengua Quechua, especialmente en la zona urbana de Cochabamba.\cite{CHLMV2014}

Se describe trabajos interdisciplinarios anteriores como:

En la gesti\'{o}n Abril 2014, LAEL lanz\'{o} una convocatoria para podcast en Audio Frances, Audio Ingles, Audio Quechua y 
Video Quechua. De forma que se fortalecio la secci\'{o}n de actividades, subscripci\'{o}n, Visualizaci\'{o}n Gr\'{a}fica .

El equipo de Frances conformado por cinco adscritos de LAEL propusieron un producto dirigido a losestudiantes de la 
Carrera de Turismo y público en general.

El presente proyecto se enfoca en la elaboraci\'{o}n y producci\'{o}n de podcast en franc\'{e}spara nivel b\'{a}sico,
tomando en cuenta el contexto cochabambino, orientado a un aprendizaje autorregulado para estudiantes, profesionales y
personas vinculadas al '{a}mbito del turismo. \cite{CMNPZ2015}

El equipo de Ingl\'{e}s conformado por seis adscritas de LAEL propusieron un producto para los estudiantes del
Centro de Interacci\'{o}n Ling\"{u}ica y p\'{u}blico en general.

La computadora es una herramienta pedag\'{o}gica que se ha vuelto m\'{a}s accesible y la WEB 2.0 facilita el aprendizaje de lenguas.
Estas herramientas modelan un cambio en la concepci\'{o}n tradicional del aula. Tambi\'{e}n, le permite al estudiante desarrollar
un aprendizaje autorregulado. El podcast es un recurso de audio en formato de Mp3 y accesible p\'{u}blica en la red. Estas grabaciones
pueden seguir un gui\'{o}n o ser improvisadas; tambi\'{e}n existen Podcast que integran audio, im\'{a}genes y comentarios.\cite{AFSTVV2015}

El equipo de Audio Quechua conformado por tres adscritas de LAEL propuso el producto enfocado en el \'{a}rea de Medicina y
Comunicaci\'{o}n Social. 

En este marco, los funcionarios p\'{u}blicos, al estar en contextos urbanos, tienen acceso casi cotidiano a las nuevas tecnolog\'{i}as
por lo que es pertinente pensar en propuestas pedag\'{o}gicas como los Podcast para el aprendizaje autorregulado del quechua. Puesto que,
al estar en l\'{i}nea mediante el internet estos podr\'{i}an acceder al material educativo sin restricciones, en el tiempo que dispusiesen,
asi como tambi\'{e}n de manera gratuita. Por tanto, este ser\'{i}a una forma de motivar a los funcionarios para que se acerquen a la lengua
quechua, lo cual beneficiar\'{i}a a ambas poblaciones, quechua y castellano, ya que aportar\'{i}a al desarrollo m\'{a}s eficaz de las interacciones comunicativas de estos.\cite{CCZ2015}

El equipo de Video Quechua conformado por tres adscritas de LAEL propuso un producto enfocado para los estudiantes de la Carrera
LAEL basado en la fon\'{e}tica del idioma.

Este trabajo es\'{a} dirigido a la producci\'{o}n y elaboraci\'{o}n de recursos multimedia (podcast-video), para el desarrollo fon\'{e}tico y la discriminaci\'{o}n auditiva de la lengua quechua. Considerando que no existe material de este tipo en la lengua quechua, se propone una producci\'{o}n de esta naturaleza por varias razones.\cite{CGL2015}

\section{Definici\'{o}n del Problema}
Actualmente LAEL carecen de soporte en el área de Tecnologías de la Información y Comunicación TICs enfocado la enseñanza
debido a que no tienen materias curriculares, ya que la actualización es por cuenta propia. En la Facultad de Humanidades 
se cuenta con el área de Unidad Técnica de Información (UTI), la misma se encarga de funciones: Control de Inventario de 
Activos Fijos,Mantenimiento Preventivo–Correctivo de Equipos de Computación u otros dispositivos electrónicos, Servicio de Red,
Soporte al Usuario (Microsoft office), Brindar Servicio Web página Facultativa, Gestión de Kardex. En general se ocupan de 
soporte administrativo. Los diferentes materiales educativos producidos por los diferentes Estudiantes de LAEL se encuentran
en estado analógico debido a su falta de un medio de difusión, quiere decir que permanece en estantes, bibliotecas y otros. 
Lo cual limita al acceso para los usuarios para quienes se desarrolló, muchos de ellos desconocidos por la sociedad.
Haciendo hincapié que la educación tradicional que por sus buenos resultados en la formación de profesionales en el área de 
la enseñanza de lenguas

Algunos Docentes de LAEL debido a su carencia de tiempo o interés en el conocimiento de nuevas herramientas las cuales logren
apoyar en la Educación Superior tradicional es tomando como marco de referencia más por sus buenos resultados.

Por lo mencionado anteriormente se define el problema como:

Escasa difusi\'{o}n de \textbf{recursos multimedia educativos} producidos por la Carrera de Lingüística Aplicada a la Enseñanza 
de Leguas dificulta el desarrollo del \textbf{aprendizaje autorregulado de las lenguas}.

\section{Objetivos}

\subsection{Objetivo General}

Contribuir con el \textbf{servicio agregado de noticias de Podcast} al fortalecimiento
del \textbf{aprendizaje autorregulado de lenguas} mediante el desarrollo de una
Plataforma Web Educativa.

\subsection{Objetivo Especifico}

\begin{itemize}

\item Proveer personalización de servicio agregador de noticias por programa de
aprendizaje (sub-categoría)

\item Implementar mecanismos de transcripción de contenido

\item Proveer representación de micro formatos para transcripción de contenido

\item Facilitar pruebas de servicio agregador de noticias, reproducción de Audio,
reproducción de Video.

\end{itemize}

\section{Justificación}
Implementar un mecanismo que coadyuve en el proceso de aprendizaje de lenguas
para un aprendizaje autorregulado.

Se realizara el proyecto con beneficio a los cibernautas que puedan completar su
habilidad auditiva y visual por medio de los Podcast.

La implementación del proyecto será realizado con tecnologías libres debido que se
trata de un proyecto de Adscripción nombrando como Unidad Patrocinadora es una
Institución Publica abocada en la formación de profesionales en el área de la
enseñanza y aprendizaje de lenguas.

\section{Alcance}

Se tendrán las siguientes áreas vistas dentro del proyecto:

\begin{itemize}

\item Gestión Servicio Agregador de Noticias
\item Animación de Transcripción de Podcast Audio.
\item Reporte de Pruebas

\end{itemize}