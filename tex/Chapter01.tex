\chapter{INTRODUCCION}

\section{Introducción}

La Carrera de Lingüística Aplicada a la Enseñanza de Lenguas (LAEL) de la Universidad Mayor de
San Simon (UMSS) forma recursos a nivel personal acorde a su medio, proponiendo mecanismos para
la enseñanza y aprendizaje de lenguas. A mediados de la gestión 2013 se elaboró material 
educativo enfocado en habilidades comunicativas como ser: hablar, escribir, leer, pensar.
Haciendo hincapié en el auditivo por medio de recursos multimedia (Podcast) educativos 
a raíz de un análisis de necesidades a funcionarios públicos y/o privados de la urbe de
Cochabamba. Los estudiantes elaboraron recursos multimedia educativos enfocados en el 
aprendizaje autorregulado de la lengua Quechua.

Se propone proveer soporte tecnológico utilizando la difusión de canales de noticias de
Podcast basados en la liberación de cada Contenido (Episodio) para el usuario 
cibernauta pueda estar actualizado sobre el Programa de Aprendizaje, realizando una 
suscripción por lengua de interés.Con apoyo de las Tecnologías de la Información y
Comunicación enfocado en la enseñanza se pretende apoyar el proceso de Aprendizaje
Autorregulado de una lengua.

\section{Antecedentes}

\scriptsize

Cualquier sitio Web es un espacio de información integrado. En muchos casos, sin embargo, este
espacio de información está de archivos HTML. Nos referimos a la ``arquitectura'' de la información
en lugar de la ``estructura'' o ``organización'' de la información con el fin de hacer hincapié en el
hecho de la estructura da forma el análisis de los requisitos funcionales del entorno. Para los
ambientes de aprendizaje, los requisitos funcionales son numerosos y no se han estudiado todavía
sistemáticamente.

Una plataforma virtual flexible será aquella que permita adaptarse a las necesidades de los alumnos
y profesores (borrar, ocultar, adaptar las distintas herramientas que ofrece); intuitivo, si su 
interfaz es familiar y presenta una funcionalidad fácilmente reconocible y, por último, amigable,
si es fácil de utilizar y ofrecer una navegabilidad clara y homogénea en todas sus páginas.

Internet se ha convertido hoy en día en el principal medio para publicar y difundir recursos e 
información en general. Podemos encontrar infinidad de recursos e información relevante para el 
desarrollo del proceso de enseñanza- aprendizaje destinado al profesorado y al alumnado. Sin embargo,
en esta categoría solo incluimos aquellos recursos que no son susceptibles de modificación y/o
publicación por parte de los usuarios, ya sean profesores o alumnos, en tanto que estos recursos serán
incluidos en la categoría de herramientas para la colaboración en red. Estos recursos serían recursos
web hipertextuales, generalmente páginas web y recursos para la docencia diseñados con aplicaciones
específicas, y bases de datos simulaciones, portales educativos y/o plataformas específicas de acceso a
información educativa y webquests. 

\normalsize

El podcast es uno de los recursos multimedia que ofrece la tecnología de hoy en día. Además, es una 
manera interactiva de aprendizaje que motiva y facilita los procesos de enseñanza y aprendizaje de una
segunda lengua. Por un lado, en la elaboración de los podcast se combinan los formatos: el texto, la imagen
y el sonido. Por otro lado, el uso de unos recursos multimedia (podcast) en la educación debe ser acorde a
la necesidad del aprendizaje o bien a los que exija la población estudiantes. Así también, Stanley define
el podcast de la siguiente manera:El podcast es un recurso disponible en la Internet, utilizado para crear
grabaciones de audio y hacerlas públicas en la red, la página principal del podcast tiene apariencia y puede
funcionar, como un blog. Además de los archivos de audio, se pueden agregar imágenes y comentarios. Los 
archivos de audio, una vez en la red, pueden funcionar como “radio” y ser descargados a computadoras 
personales, así como a CD o aparatos portátiles (MP3 o MP4) y ser escuchados tantas veces como sea de interés
del oyente en el sitio y hora de su conveniencia. Puede ser aplicado para cualquier
idioma

\scriptsize

\blockquote{
En general, los estudiantes se pueden describirse como autorregulado al grado que son meta cognitiva, 
'motivacional', y participantes conductivamente activos de su propio proceso de aprendizaje. Estos estudiantes personalmente inician y dirigen sus propios esfuerzos para adquirir conocimiento y habilidades en lugar de
confiar en los profesores, padres, u otros agentes de instrucciones. Para calificar específicamente como 
autorregulado en mi cuenta, los estudiantes deben incluir el uso de determinadas estrategias para alcanzar
metas académicas sobre la base del auto eficacia de perfecciones. Esta definición asume la importancia de tres
elementos: estrategias estudiantes de autorregulación del aprendizaje, las percepciones de auto eficacia de
rendimiento, habilidad y compromiso para metas académicas. Estrategias de aprendizaje autorregulado son 
acciones y procesos dirigidos a la adquisición de información o habilidad que involucra a la agencia, el
propósito y las percepciones instrumentales por alumnos. Estos incluyen métodos tales como la organización y la
transformación de la información, la búsqueda de información y ensayando o utilizando ayudas memoria \cite{zimmerman1989social}}

\normalsize 

Se describe trabajos similares como: 

La Carrera LAEL impulsó en la creación de material multimedia bajo un estudio de necesidad de funcionarios 
públicos y/o privados en tener comunicación con personas quechua hablantes que migraron para contar con
beneficios como ser: hospital, colegio, juzgado, mercado de abasto, servicio de identificación personal, 
contextos dentro el área urbana de la ciudad de Cochabamba como producto salió una tesis en el año 2013 
:“Elaboración y Producción de Podcast para el aprendizaje Autorregulado de la Lengua Quechua”.\cite{HVLMC2013}

\section{Definición del Problema}

Actualmente LAEL carecen de soporte en el área de Tecnologías de la Información y
Comunicación (TIC's) enfocado la enseñanza debido a que no tienen materias
curriculares, ya que la actualización es por cuenta propia.
En la Facultad de Humanidades se cuenta con el área de Unidad Técnica de
Información (UTI), la misma se encarga de funciones: Control de Inventario de
Activos Fijos, Mantenimiento Preventivo–Correctivo de Equipos de Computación u
otros dispositivos electrónicos, Servicio de Red, Soporte al Usuario (Microsoft
office), Brindar Servicio Web página Facultativa, Gestión de Kardex. En general se
ocupan de soporte administrativo.
Los diferentes materiales educativos producidos por los diferentes Estudiantes de
Lingüística se encuentran en estado analógico debido a su falta de un medio de
difusión, quiere decir que permanece en estantes, bibliotecas. Lo cual limita al acceso
para los usuarios para quienes se desarrolló, muchos de ellos desconocidos por la
sociedad.
Haciendo hincapié que la educación tradicional que por sus buenos resultados en la
formación de profesionales en el área de la enseñanza de lenguas

Algunos Docentes de LAEL debido a su carencia de tiempo o interés en el
conocimiento de nuevas herramientas las cuales logren apoyar en la Educación
Superior tradicional es tomando como marco de referencia más por sus buenos
resultados.

Por lo mencionado anteriormente se define el problema como:

Escasa difusión de \textbf{recursos multimedia educativos} producidos por la Carrera de
Lingüística Aplicada a la Enseñanza de Leguas dificulta el desarrollo del \textbf{aprendizaje
autorregulado de las lenguas}.

\section{Objetivos}

\subsection{Objetivo General}

Contribuir con el \textbf{servicio agregado de noticias de Podcast} al fortalecimiento
del \textbf{aprendizaje autorregulado de lenguas} mediante el desarrollo de una
Plataforma Web Educativa.

\subsection{Objetivo Especifico}

\begin{itemize}

\item Proveer personalización de servicio agregador de noticias por programa de
aprendizaje (sub-categoría)

\item Implementar mecanismos de transcripción de contenido

\item Proveer representación de micro formatos para transcripción de contenido

\item Facilitar pruebas de servicio agregador de noticias, reproducción de Audio,
reproducción de Video.

\end{itemize}

\section{Justificación}
Implementar un mecanismo que coadyuve en el proceso de aprendizaje de lenguas
para un aprendizaje autorregulado.

Se realizara el proyecto con beneficio a los cibernautas que puedan completar su
habilidad auditiva y visual por medio de los Podcast.

La implementación del proyecto será realizado con tecnologías libres debido que se
trata de un proyecto de Adscripción nombrando como Unidad Patrocinadora es una
Institución Publica abocada en la formación de profesionales en el área de la
enseñanza y aprendizaje de lenguas.

\section{Alcance}

Se tendrán las siguientes áreas vistas dentro del proyecto:

\begin{itemize}

\item Gestión Servicio Agregador de Noticias
\item Animación de Transcripción.
\item Gestión de Micro formatos
\item Reporte de Pruebas

\end{itemize}