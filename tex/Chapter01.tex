\chapter{Introducción}

Me propongo a exponer, la situación de lenguas originarias en Bolivia: Aymara,
Guaraní, Quechua, etc. brindando soporte por parte de la Ley General de Derechos y
Políticas Lingüísticas 269 \textquotedouble{promueve la implementación de una
lengua origina según la región} orientado específicamente a funcionarios públicos
y/o privados; por consiguiente la promulgación de la ley genero la demanda de
recursos educativos para el aprendizaje de una segunda lengua. Mas aún las
lenguas extranjeras: Inglés y Francés que se utilizan para situaciones de: turismo,
negocio, formación profesional entre otros.

Así mismo, se propone implementar la gestión de Podcast de tipo audio/vídeo
elaborado por Carrera de Lingüística Aplicada a la Enseñanza de Lenguas (LAEL)
brindando la creación de un subscriptor de noticia por parte de la Carrera de
Informática para notificación de disponer nuevo recurso dentro una Plataforma
Web Educativa.

Razón por la cual, se opta por utilizar la licencia LPG-Bolivia propuesta por
la ADSIB \footnote{ADSIB: Agencia para el desarrollo de la sociedad de la
información en Bolivia. \cite{LPGBolivia}}. Segun \cite{LPGBolivia} denomina 
\textquotedouble{protege los derechos intelectuales y materiales de
software en Instituciones Publicas}.


\section{Antecedentes}

En el año 2013, la Carrera LAEL promovió la creación de recursos educativos
digitales para funcionario públicos y/o privados de la zona urbana del
municipio de Cochabamba. Personas caracterizado como Quechua hablantes
necesitan el uso y demanda de servicios: salud, educación, justicia,
financiera entre otros. 

Por otro lado en el año 2014, la Carrera LAEL firmo un proyecto de adscripción
con la Carrera de Informática; cinco equipos de estudiantes de LAEL realizaron
diferentes Podcast de audio/vídeo. El equipo de Informática realizo el soporte
de difusión sobre la red de Internet para los recursos educativos digitales.

\section{Definición del Problema}

El profesional de la Carrera LAEL produce recursos educativos; por lo general
este material, permanece en textos.

Los recursos educativos se dispone en biblioteca LAEL. Por tal motivo se su
acceso es considerado de limitado para la sociedad.

Por una parte, los recursos educativos digitales para lenguas extranjeras
existen en diversidad y proviene del exterior. Por otro lado, los recursos
educativos digitales para lenguas originarias son muy escasos. 
 																						
Por lo mencionado anteriormente se define el problema como:

Escasa difusión de \textbf{recursos digitales educativos} producidos por 
la Carrera de LAEL dificulta el desarrollo del \textbf{aprendizaje 
autorregulado de las lenguas de Francés, Inglés y Quechua}.

\section{Objetivo}

\subsection{General}

Implementar un \textbf{subscriptor de Podcast} para el fortalecimiento del 
\textbf{aprendizaje autorregulado de lenguas Francés, Inglés, Quechua} mediante
el desarrollo de una Plataforma Web Educativa.

\subsection{Específicos}

\begin{itemize}

\item Proveer personalización de subscriptor Podcast.
\item Implementar un mecanismo para transcripción y glosario.
\item Proveer representación de la web semántica para transcripción y glosario.
\item Realizar una prueba unitaria para subscriptor, reproducción de Audio y
reproducción de Vídeo.

\end{itemize}

\section{Justificación}

\subsection{Tecnológica}

Plataforma Web Educativa realiza difusión de recursos educativos digitales;
los entornos virtuales de aprendizaje apoyan el fortalecimiento y enseñanza de
una disciplina (lengua).

\subsection{Social}

Recursos educativos digitales tiene la principal característica acorde a la
realidad del departamento de Cochabamba.

Los recursos educativos digitales son creados a necesidad de los usuarios,
quien desee conocer una lengua.

\subsection{Económica}

Recursos educativos digitales tiene las características: simple y gratuito.

\section{Limitaciones}

\begin{itemize}

\item Plataforma Web Educativa no brinda un módulo avanzado de usuarios, los
roles de sistema tiene acceso a privilegios.

\item Plataforma Web Educativa simplemente gestiona: Podcast, Actividades,Transcripción,
infografía; los recursos educativos digitales no brinda creación de ningún
tipo de recurso.

\item Subscriptor de noticia utiliza la version RSS 2.0 creación de canales,
los canales de noticia creados por programa de aprendizaje

\item La plataforma Web Educativa es de tipo web no debe funcionar como sistema
de escritorio.

\item Dar de baja suscripción de usuario de sistema realiza la notificación
por gestión de subscriptor.

\item Conexión a Internet disponer de un servicio por parte de un proveedor.

\end{itemize}