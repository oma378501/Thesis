\chapter*{Ficha Resumen}

La creación de Recursos Multimedia Educativos son deseos provenientes de 
\'{a}reas como ser: Ciencias de la Educaci\'{o}n, Ling\"{u}\'{i}stica dentro la
UMSS.
De tal forma que coadyuven en fortalecer habilidades comunicativas que puedan 
llegar a tener la diversidad de personas en la sociedad.

A trav\'{e}s de Ley 269 promulgada 2 Agosto 2012, por el Presidente Evo Morales
en apoyo al fortalecimiento de idiomas dentro del Estado Plurinacional de 
Bolivia da a comprender que las lenguas originarias forman parte de la cultura,
la cultura es patrimonio del estado la cual la constitución debe de proteger a
pueblos reconocidos y personas hablantes.

En la gesti\'{o}n 2014 se realiza una convocatoria para un Proyecto de 
Adscripci\'{o}n, la cual enfoca los idiomas de Ingl\'{e}s B\'{a}sico, Frances
B\'{a}sico, Quechua B\'{a}sico, Quechua Psicosocial, Fon\'{e}tica Quechua, 
adem\'{a}s de contar con la colaboraci\'{o}n de Carreras de Inform\'{a}tica y
Sistemas. Considerado como un proyecto interdisciplinario como respuesta la
difusi\'{o}n de recursos por parte del \'{a}rea de Ling\"{u}\'{i}stica 
proponiendo para la sociedad como soluci\'{o}n parcial.

Se propone implementar un Servicio de Noticias basado en Podcast que son 
elaborados por propios Adscritos de Ling\"{u}\'{i}stica el mismo que este sujeto
a subscripci\'{o}n de noticias en la liberaci\'{o}n de Podcast en Audio/Video 
valga la redundancia brindando material educativo, gratuito para los funcionarios
p\'{u}blicos puedan descargar el Podcast y escucharlo en su tiempo libre para 
reforzar un conocimiento en el idioma Quechua formando un aprendizaje 
autorregulado.

En experiencia de trabajar con personas desarrollo y otras \'{a}rea de trabajo 
se recomienta manejar estandares de trabajo, adem\'{a}s de utilizar herramientas
open source para optimizar la elaboraci\'{o}n de recursos como ser: imagen, audio,
video, historietas debido que los mismos tendran que sujetarse estar a conexiones
lentas.