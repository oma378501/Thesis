\chapter{Conclusiones y Recomendaciones}

Como consecuencia del proyecto de adscripción, se tiene las siguientes
conclusiones y recomendaciones:

\section{Conclusiones}

\begin{itemize}

\item Se provee de noticias por programa aprendizaje, para recibir
notificación de nuevo contenido vía correo electrónico.

\item Se genera un subtitulado de reproductor Podcast con su transcripción.
Además la gestión de glosario.

\item Se agrega contenido semántico sobre subtitulado y representación de
glosario.

\item Se realiza pruebas de unidad e integración; para buenas practicas en
programación.

\end{itemize}

\section{Recomendaciones}

\subsection{Técnicos}

\begin{itemize}

\item Configurar inicio de sesión externo por red social:
Facebook, Google, Twitter. Requiere una dirección pública de IP 
\footnote{IP:
Es el método o protocolo por el cual se envían datos desde un ordenador a
otro a través de Internet. Cada computadora en Internet tiene al menos una
dirección IP que identifica de forma exclusiva de todos los demás ordenadores
en Internet. \cite{ip}}; Solicitar permiso al responsable de la red de
computadoras del Centro MEMI \footnote{MEMI: Centro de Mejoramiento de la
Enseñanza Matemática e Informática} Área Informática (Ing. Jorge Orellana). 

\item Servidor web de producción, tener un miembro dentro el equipo de
desarrollo (Rudy Rojas) quien disponga de acceso remoto un servidor externo; como
ambiente para tareas segundo plano y servicio de correo.

\item Un servidor web de producción por parte de la unidad Patrocinadora
(Carrera LAEL), solicitar autorización por escrito al responsable (Lic. Mario Antezana).

\item Extractor de micro-formatos \footnote{micro-formato: http://pin13.net/},
tener acceso a un equipo que disponga de una dirección pública de IP.

\item Web semántica, utilizar chat en linea IRC
\footnote{IRC: irc://irc.freenode.net/microformats} para compartir consejos
de terceros.

\item Sistema de tipo web, utilizar una distribución Linux
como entorno desarrollo; para realizar transferencia de tecnología.

\item Framework \footnote{framework: Es un conjunto
de recursos y herramientas para desarrolla-dores de software para crear y
gestionar aplicaciones web, servicios web y sitios web. \cite{framework}},
considerar aspectos: cantidad miembros en la comunidad, curva de aprendizaje.

\item Framework como estándar de trabajo, contempla documentación para lo cual
requiere inversión de tiempo;

\item Gestor de base de datos (relacional), verificar generación de llaves
primarias compuestas, incorporación de módulo de pruebas.

\item Estándares de trabajo dentro el equipo desarrollo: definir políticas
internas, convención de variables, nombre de función; para ello elaborar
un documento de especificacion.

\item Herramientas colaborativas, utilizar control de version de código (git),
manejado-res de tarea (pivotal tracker).

\item Habilidades ágiles en equipo desarrollo: disciplina, responsabilidad,
honestidad;

\item Licencia LPG-Bolivia tiene ventajas en proyectos de adscripción
(Anexo \ref{chap:LPG-Bolivia}).

\end{itemize}

\subsection{Trabajo Multidisciplinario}

\begin{itemize}

\item Términos de Referencia, verificar la existencia de este documento, caso
contrario realizar entrevistas con el coordinador de proyecto para elaboración.

\item Reunión de presentación de autoridades: coordinador, tutor, directora de
Informática, directora de LAEL.  

\item Adscrito de Informática, debe mantener buena comunicación con el
coordinador.

\item Adscrito de Informática, representa su unidad de origen dentro un proyecto
de adscripción; de tal motivo practicar: responsabilidad, disciplina y
honestidad.

\item Seguimiento del proyecto, contemplar respaldado de documentación: adscrito,
coordinador. finalizando los requerimientos, solicitar carta conclusión de
proyecto.

\item Extravió de documento de aceptación, firmar la recepción del documento de
entrega entre el representante del equipo desarrollo (Omar Huanca) y
coordinador (Lic Manuel Camacho).

\item El tutor de Informática, debe ser docente a tiempo completo; de tal motivo 
puede solicitar programar una reunión.

\item Equipo desarrollo, si se tiene uno o mas adscritos de Informática, elegir
a un solo docente para la definición de áreas de trabajo.   

\end{itemize}

\subsection{Redacción}

\begin{itemize} 

\item Editor de texto, utilizar un editor de texto de mayor conocimiento, caso
contrario aprender otro editor (\LaTeX) contempla un tiempo de aprendizaje.

\item Un documento de tesis, debe contemplar reglas de gramática, genero,
cardinal entre otros; acudir al área de LAEL para solicitar colaboración.

\item Terminado la redacción de este Capítulo re-estructurar ideas del
Capítulo 1 (Introducción).

\end{itemize}

\section{Trabajos Futuros}

Se recomienda las siguientes experiencias para proyectos similares:

\begin{itemize}

\item Si se trata de un proyecto digital se sugiere utilizar servidores web 
especializados como ser: nginx \footnote{ngnix: Es conocido por su alto 
rendimiento, la estabilidad, la gran variedad de funciones, configuración
simple, y bajo consumo de recursos. \cite{nginx}}  o tal vez lighttpd. \footnote{lighttpd: 
Esta diseñado y optimizado para entornos de alto rendimiento. Con una 
pequeña huella de memoria en comparación con otros servidores web, la
gestión eficaz de la CPU de carga y avanzando lighttpd conjunto de 
características es la solución perfecta para cada servidor que está
sufriendo problemas de carga. \cite{lighttps}}
 
\item Si se desea crear material digital como imágenes, historietas, producción
de audio/vídeo es aconsejable utilizar herramientas denominadas open source 
\footnote{open source: Es software cuyo código fuente está disponible 
para su modificación o mejora por parte de nadie. \cite{openSource}} debido
que estos recursos tienen que ser usados para difusión, tomando en cuentas las
características propias de un software open source, como ser: multi-plataforma y
optimizar recursos para realizar transmisión en un ancho de banda limitado.

\end{itemize}