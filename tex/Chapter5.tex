\chapter{Conclusiones y Recomendaciones}

Como consecuencia del proyecto de adscripción, se tiene las siguientes
conclusiones y recomendaciones.

\section{Conclusiones}

\begin{itemize}

\item Se provee de un servicio de suscripción de noticia podcast vía correo
electrónico, el cual también permite la opción al usuario de darse de baja.

\item Se sincroniza transcripción con reproductor podcast por medio de un
subtitulado.

\item Se proporciona la gestión de glosario de podcast.

\item Se agrega contenido semántico para subtitulado y representación de
glosario.

\item Se utiliza pruebas de unidad para agregar manejo de una situación no 
esperada.

\end{itemize}

\section{Recomendaciones}

\subsection{Técnicas}

\begin{itemize}

\item La configuración de inicio de sesión externo por red social requiere
de una dirección pública de IP \footnote{IP: Es el método o protocolo por el cual
se envían datos desde un ordenador a otro a través de Internet. Cada
computadora en Internet tiene al menos una dirección IP que identifica de
forma exclusiva de todos los demás ordenadores en Internet. \cite{ip}}.

\item El servidor web de producción es requerido  por parte de la unidad
patrocinadora, solicitar autorización por escrito al responsable para
la transferencia de tecnología.

\item El proyecto de adscripción de software debe utilizar licencia LPG-Bolivia,
descrito en Anexo \ref{chap:LPG-Bolivia}.

\end{itemize}

\subsection{Herramientas}

\begin{itemize}

\item El el uso de un framework \footnote{framework: Es un conjunto de
recursos y herramientas para desarrolladores de software para crear y
gestionar aplicaciones web, servicios web y sitios web. \cite{framework}}
contempla los siguientes aspectos miembros de la comunidad y curva de
aprendizaje.

\item El uso de un extractor de microformato \footnote{microformato: http://pin13.net/}
requiere disponer de un equipo con dirección pública de IP.

\item Se recomienda utilizar un canal IRC 
\footnote{IRC: irc://irc.freenode.net/microformats} como medio de comunicación
para solicitar sugerencia y uso de microformato.

\item El uso de herramientas colaborativas de control de versión de código
(git) y manejadores de tarea (pivotal tracker) ofrece soporte de trabajo
entre personas.

\item Utilizar un servidor web de producción como ambiente de pruebas para tareas
de segundo plano y servicio de correo.

\end{itemize}

\subsection{Postular a proyecto de adscripción}

\begin{itemize}

\item Los términos de referencia deben ser elaborados por la unidad
patrocinadora, caso contrario el equipo de informática tiene que realizar
entrevistas con el responsable del proyecto.

\end{itemize}

\subsection{Proceso proyecto de adscripción}

\begin{itemize}

\item Las autoridades de un proyecto deben realizar una reunión de
presentación, Además de dar a conocer el estado de avance.

\item El equipo de desarrollo debe disponer de una ejemplar original de
documentos de términos de referencia, historias de usuario, documento de
validación y aceptación.

\item El tutor y el equipo de desarrollo deben disponer de un ambiente de
trabajo.

\end{itemize}

\subsection{Conclusión proyecto de adscripción}

\begin{itemize}

\item Equipo desarrollo debe solicitar una carta de conclusión de proyecto
dirigida la autoridad responsable de la unidad patrocinadora adjuntando
copia de los documentos de validación y aceptación.

\end{itemize}

\subsection{Redacción}

\begin{itemize} 

\item Un documento de redacción de adscripción debe contemplar reglas
gramaticales de ortografía, coherencia y concordancia por lo cual se debe
solicitar colaboración a una persona del área de LAEL.

\item Concluida la redacción del Capítulo 5 (Conclusiones y Recomendaciones),
se recomienda estructurar el Capítulo 1 (Introducción).

\item Las figuras de elaboración propia deben ser escritas en lenguaje 
español.

\end{itemize}

\subsection{Proceso de trabajo en equipo desarrollo}

\begin{itemize}

\item El equipo de desarrollo debe utilizar las habilidades de disciplina,
responsabilidad y honestidad.

\item El equipo de desarrollo debe utilizar estándares de trabajo de convención
de variable, nombre de la función, lenguaje a utilizar y herramientas de
soporte de manera que se elabore un documento como reglamento interno.

\end{itemize}

\section{Trabajos futuros}

Se recomienda las siguientes experiencias para proyectos similares.

\begin{itemize}

\item Un proyecto de adscripción es considerado como un proyecto de grado
por contemplar de un cliente real, aplicación de conocimiento en ciencias de
la computación y propuesta de solución a un problema específico.

\item Si se trata de un proyecto digital se sugiere utilizar servidores web 
especializados como nginx \footnote{ngnix: Es conocido por su alto 
rendimiento, la estabilidad, la gran variedad de funciones, configuración
simple, y bajo consumo de recursos. \cite{nginx}}  o lighttpd \footnote{lighttpd: 
Esta diseñado y optimizado para entornos de alto rendimiento. Con una 
pequeña huella de memoria en comparación con otros servidores web, la
gestión eficaz de la CPU de carga y avanzando lighttpd conjunto de 
características es la solución perfecta para cada servidor que está
sufriendo problemas de carga. \cite{lighttps}}.
 
\item Crear material digital como de imagen, historieta y producción de
audio/vídeo utilizar software open source \footnote{open source: Es software
cuyo código fuente está disponible para su modificación o mejora por parte
de nadie. \cite{openSource}}, el software open source brinda las siguientes
características de multiplataforma y optimiza recursos para realizar
transmisión para ancho de banda limitado.

\end{itemize}