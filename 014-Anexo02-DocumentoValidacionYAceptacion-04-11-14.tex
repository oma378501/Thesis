\chapter{Documento de validación y aceptación 04-11-2014}

El presente documento tiene por finalidad el de constatar la conformidad tanto
del cliente como el equipo de desarrollo.

Esta es la presentación de la versión alpha correspondiente al sprint primero
\textquotedblleft versión alpha listo\textquotedblright.

Dicha presentación está sujeto al 17 (por ciento) del total de avance del
sistema, para mayor conformidad para ambas partes se remarca los
siguientes aspectos.

\begin{enumerate}

\item \textbf{Escala de evaluación}

\begin{table}[!h]
\centering
\begin{tabular}{|c|c|c|}
\hline
\textbf{Punta je} & \textbf{Valor Numérico} & \textbf{Concepto} \\ \hline
NA & 0 & \begin{tabular}[c]{@{}c@{}}Se necesita hacer cambios\\ profundos\end{tabular} \\ \hline
RA & 1 & \begin{tabular}[c]{@{}c@{}}Requiere modificación para\\ aceptar\end{tabular} \\ \hline
A & 2 & Aceptado \\ \hline
\end{tabular}
\captionof{table}{Escala de evaluación}
\end{table}

\item \textbf{Las funcionalidades a presentar son las siguientes:}

\begin{itemize}

	\item Registro manual de usuario aprendiz a la plataforma.
		\begin{itemize}
			\item Envió de solicitud de confirmación de correo electrónico a bandeja de entrada.
			\item Modificar datos personales.
		\end{itemize}
	\item Registro por medio de red social.
		\begin{itemize}
			\item Registro por Facebook.
			\item Registro por Google.
			\item Registro por Twitter.
		\end{itemize}
	\item Implementar Página Maestra de Plataforma.

\end{itemize}

\item El pago del 17 (por ciento) se dará por hecho si se está conforme 
con 3 de las 4 funcionalidades.

\end{enumerate}

Se está conforme con una funcionalidad si se cumple los siguientes aspectos
estipulados en el sumario de evaluación de párrafos.

\section{Sumario de evaluación de parámetro}

\begin{itemize}

\item \textbf{Registro manual de usuario aprendiz a la plataforma}

\begin{itemize}

\item \textbf{Envió de solicitud de confirmación de correo electrónico a 
bandeja de entrada}

Envió de solicitud de confirmación de correo electrónico a bandeja
de entrada. Abra un navegador web con la siguiente dirección (URL): plataforma
\footnote{plataforma: http:// plataformaeducativalael.hum.umss.edu.bo}, pinche en 
el enlace Registrar ubicado en la parte superior derecha.

\begin{table}[!ht]\centering
\begin{tabular}{|l|l|l|}
\hline
\multicolumn{1}{|c|}{\textbf{N}} & \multicolumn{1}{c|}{\textbf{Parametros a evaluar}} & \multicolumn{1}{c|}{\textbf{Puntaje}} \\ \hline
01 & \begin{tabular}[c]{@{}l@{}}\textquestiondown Ud. puede ingresar datos en los campos y realizar clic en el botón\\ Registrar: Dirección de Correo, Nombre de Usuario, Contraseña, Repetir\\ Contraseña, Seleccione un valor de Ocupación, Sub Ocupación, Captcha?\end{tabular} &  \\ \hline
02 & \begin{tabular}[c]{@{}l@{}}\textquestiondown Ud. puede leer un mensaje \textquotedouble{Un correo electrónico que contiene m\'{a}s\\ instrucciones ha sido enviada bandeja de dirección de correo electrónico\\ proveedor}?\end{tabular} &  \\ \hline
03 & \begin{tabular}[c]{@{}l@{}}\textquestiondown Ud. ingrese a su cuenta de correo con la cual se registró anteriormente,\\ revise en bandeja spam, un mensaje de Administrador en el cuerpo un\\ enlace de confirmación?\end{tabular} &  \\ \hline
04 & \begin{tabular}[c]{@{}l@{}}\textquestiondown Ud. puede pinchar en el enlace y mostrarle el siguiente mensaje \textquotedouble{Su\\ dirección de correo electrónico ha sido verificado exitosa mente}, en la\\ parte superior de inicio sesión?\end{tabular} &  \\ \hline
\end{tabular}
\captionof{table}{Envió de solicitud de confirmación de correo electrónico}
\end{table}

Nota: Si está conforme con la funcionalidad \textquotedouble{Envió de
solicitud de confirmación de correo electrónico a bandeja de entrada}
si la sumatoria de Puntaje tiene un valor mayor y/o igual 5.

Las siguientes funcionalidades se realizan con credenciales propias.

\item \textbf{Modificar datos personales}

En la presentación principal en la parte superior derecha se tiene un enlace 
Inicio Sesión, pinchar sobre el mismo y mostrara una ventana de autentificar
con los siguientes campos: Nombre de Usuario, Contraseña.

\begin{table}[H]\centering
\begin{tabular}{|l|l|l|}
\hline
\multicolumn{1}{|c|}{\textbf{N}} & \multicolumn{1}{c|}{\textbf{Parámetros a evaluar}} & \multicolumn{1}{c|}{\textbf{Puntaje}} \\ \hline
01 & \begin{tabular}[c]{@{}l@{}}\textquestiondown Ud. pude visualizar un Panel Administrativo en donde si tiene un mensaje\\ \textquotedouble{Hola} acompañado de su nombre de usuario en la parte superior derecha?\end{tabular} &  \\ \hline
02 & \begin{tabular}[c]{@{}l@{}}\textquestiondown Ud. puede visualizar un sus datos personales pinchando sobre la imagen\\ de una persona ubicada en la esquina superior derecha y presionar la\\ opción Perfil?\end{tabular} &  \\ \hline
03 & \begin{tabular}[c]{@{}l@{}}\textquestiondown Ud. puede editar sus datos personales pinchando en el enlace \textquotedouble{Actualizar\\ Usuario} ubicado en la parte inferior\end{tabular} &  \\ \hline
04 & \begin{tabular}[c]{@{}l@{}}\textquestiondown Ud. puede salvar sus datos pinchando en el bot\'{o}n \textquotedouble{Salvar} sobre el\\ c\'{o}digo verificación?\end{tabular} &  \\ \hline
05 & \begin{tabular}[c]{@{}l@{}}\textquestiondown Ud. puede cerrar sesión haciendo clic en la imagen del hombre cito\\ ubicado en la parte superior derecha del Panel Administrativo la opción\\ Cerrar Sesión?\end{tabular} &  \\ \hline
\end{tabular}
\captionof{table}{Modificar datos personales}
\end{table}

Nota: Si está conforme con la funcionalidad \textquotedouble{Modificar 
datos personales} si la sumatoria de Puntaje tiene un 
valor igual o mayor 6.
	
\end{itemize}

\end{itemize}

\section{Registro por medio de red social}

En la presentación principal en la parte superior derecha se tiene un enlace 
Inicio Sesión, pinchar sobre el mismo y mostrara una ventana de autentificar
con los siguientes enlaces:

\begin{itemize}
	\item Ingresar con Google.
	\item Ingresar con Facebook.
	\item Ingresar con Twitter.
\end{itemize}

Tomar en cuenta que para esta funcionalidad de registro por medio de red social
es con una cuenta de correo utilizada por primera vez en el sistema.

\begin{table}[H]\centering
\begin{tabular}{|l|l|l|}
\hline
\multicolumn{1}{|c|}{\textbf{N}} & \multicolumn{1}{c|}{\textbf{Parametros a evaluar}} & \multicolumn{1}{c|}{\textbf{Puntaje}} \\ \hline
01 & \begin{tabular}[c]{@{}l@{}}\textquestiondown Ud. puede Registrarse pinchando sobre el bot\'{o}n Ingresar con Google\\ llenar con sus credenciales y un evento Login, le muestra un Panel\\ Administrativo donde en la parte superior derecha superior se muestra\\ su correo electrónico?\end{tabular} &  \\ \hline
02 & \begin{tabular}[c]{@{}l@{}}\textquestiondown Ud. puede cerrar sesión haciendo clic en la imagen del hombre cito\\ ubicado en la parte superior derecha del Panel Administrativo la opción\\ Cerrar Sesión?\end{tabular} &  \\ \hline
03 & \begin{tabular}[c]{@{}l@{}}\textquestiondown Ud. puede Registrarse pinchando sobre el bot\'{o}n Ingresar con Facebook\\ y llenando las credenciales de la red social y pinchando en el bot\'{o}n Login,\\ entonces le muestra un Panel Administrativo donde en la parte superior\\ derecha superior se muestra su correo electrónico?\end{tabular} &  \\ \hline
04 & \begin{tabular}[c]{@{}l@{}}\textquestiondown Ud. puede cerrar sesión haciendo clic en la imagen del hombre cito\\ ubicado en la parte superior derecha del Panel Administrativo la opción\\ Cerrar Sesión?\end{tabular} &  \\ \hline
05 & \begin{tabular}[c]{@{}l@{}}\textquestiondown Ud. puede Registrarse pinchando sobre el bot\'{o}n Ingresar con Twitter\\ llenar con sus credenciales y un evento Sign in luego dándole permiso de\\ autorización, le muestra un Panel Administrativo donde en la parte superior\\ derecha superior se muestra su correo electrónico?\end{tabular} &  \\ \hline
06 & \begin{tabular}[c]{@{}l@{}}\textquestiondown Ud. puede cerrar sesión haciendo clic en la imagen del hombre cito\\ ubicado en la parte superior derecha del Panel Administrativo la opción\\ Cerrar Sesión?\end{tabular} &  \\ \hline
\end{tabular}
\captionof{table}{Registro por medio de red social}
\end{table}

Nota: Si está conforme con la funcionalidad \textquotedouble{Registro por 
medio de red social} si la sumatoria de Puntaje tiene un
valor mayor igual 6.

\section{Implementar página maestra de plataforma}

\begin{table}[!ht]\centering
\begin{tabular}{|l|l|l|}
\hline
\multicolumn{1}{|c|}{\textbf{N}} & \multicolumn{1}{c|}{\textbf{Parametros a evaluar}} & \multicolumn{1}{c|}{\textbf{Puntaje}} \\ \hline
01 & \begin{tabular}[c]{@{}l@{}}\textquestiondown Ud. puede reproducir un vídeo pinchando sobre la imagen \textquotedouble{VER\\ PRESENTACIÓN} ubicada en la parte superior de la Página Maestra?\end{tabular} &  \\ \hline
02 & \begin{tabular}[c]{@{}l@{}}\textquestiondown Ud. puede Registrarse pinchando sobre el bot\'{o}n Ingresar con Google\\ llenar con sus credenciales y un evento Login, le muestra un Panel\\ Administrativo donde en la parte inferior de Inicio, se muestra su correo\\ electrónico?\end{tabular} &  \\ \hline
\end{tabular}
\captionof{table}{Implementar página maestra de plataforma}
\end{table}

Nota: Si está conforme con la funcionalidad \textquotedouble{Implementar 
Página Maestra de Plataforma} si la sumatoria de Puntaje tiene un valor 2.

Si se cumple el punto 3 procedemos a firmar este documento los representantes
para su aceptación.

\begin{table}[!ht]\centering
\begin{tabular}{|l|l|}
\hline
\begin{tabular}[c]{@{}l@{}}..........................................\\ Lic. Manuel Camacho Arce\\ PRODUCT OWNER\end{tabular} & \begin{tabular}[c]{@{}l@{}}.........................................\\ Juan Omar Huanca Balboa\\ SCRUM MASTER\end{tabular} \\ \hline
\end{tabular}
\captionof{table}{Comprobante de documento de aceptación}
\end{table}