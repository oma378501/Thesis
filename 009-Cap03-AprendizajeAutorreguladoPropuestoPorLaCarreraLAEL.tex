\chapter{Aprendizaje Autorregulado propuesto por la Carrera LAEL}

\section{Autorregulado y aprendizaje autorregulado}

El aprendiz autorregulado se denomina a toda persona quien invierte su
tiempo en mejorar su conocimiento respecto a una lengua. Esta persona
realiza una planificación propia y autosugestiona su aprendizaje.

Ahora veamos lo que menciona Schunk(1997) la autorregulación en los
pensamientos, sentimientos y actos originados por los estudiantes que están
orientados sistemáticamente a la consecución de metas. Finalmente Zimmerman
(1994) define como el grado en el cual los individuos son participantes
activos en su propio proceso de aprendizaje desde el punto de vista
metacognitivo, motivación y relativo a su comportamiento. \cite{rinconconsideraciones}

\section{Descripción de la unidad patrocinadora}

La Carrera de LAEL lleva 35 años de trayectoria al
interior de la Facultad de Humanidades y Ciencias de la Educación, tuvo un
recorrido histórico, teniendo en cuenta las raíces que obtuvo
para su surgimiento en el año 1972, brindando materias de servicio dentro la
Facultad de Ciencias Puras y Naturales, en la Universidad Mayor de San Simón.
\cite{CMNPZ2014}

Actualmente en la Carrera de LAEL brinda servicios de enseñanza de lenguas,
traducción de documentos, planificación para las lenguas originarias y
extranjeras.

\subsection{Perfil profesional del estudiante en la Carrera LAEL}

En el año 2009, el estudiante de la Carrera LAEL, destacó el perfil
profesional constituido textualmente de la siguiente manera. \cite{Q2014}

\begin{itemize}

\item Un profesional comprometido con su medio en el que gracias a procesos de
investigación de la realidad boliviana aplicará métodos y técnicas
adecuados dentro del proceso de enseñanza aprendizaje de lenguas en el 
sistema educativo nacional y universitario.

\item Será capaz de evaluar y adaptar métodos de enseñanza, tanto para
las lenguas extranjeras como para el castellano y el quechua: lengua extranjera
y/o segunda lengua.

\item Realizar investigación multidisciplinaria para estudio e 
interpretación sobre:

	\begin{itemize}
		
	\item La enseñanza de las lenguas en el sistema educativo.
	\item Problemas de alfabetización en nuestro país, aportando desde la
	perspectiva de las lenguas.
	\item Problemas específicos de bilingüismo y de las relaciones entre 
	la lengua materna y la segunda lengua.
	\item Característica del castellano boliviano en sus diferentes niveles
	culturales.
	
	\end{itemize}
	
\item Investigar sobre las lenguas, realizando estudios comparativos de sistemas
de comunicación y estructuras de las lenguas en todos los niveles de 
enseñanza.

\item Evaluador de contenidos de las asignaturas relacionadas con el área de
lenguas en todos los niveles de enseñanza.

\item Desempeñar eficientemente en cualquier otro campo en el que exige 
conocimiento y formación de lenguas (Documento Carrera Lingüística 
Aplicada a la Enseñanza de lenguas 2009).

\end{itemize}

El perfil profesional del estudiante de la Carrera LAEL, señala que los
objetivos están enfocados en su mayoría en el área de técnicas que corroboran
al ámbito educativo a través de la enseñanza de las lenguas (L1 y L2). Este
documento esta vigente. \cite{Q2014}

\subsection{Objetivos profesionales}

A continuación se da a conocer los objetivos. EL licenciado de LAEL será capaz
de:

\begin{itemize}

\item Interpretar la actualidad de la educación nacional, particularmente en
la lingüística, proponiendo metodologías específicas para la enseñanza de la
lengua originaria y/o extranjera.
 
\item Analizar e interpretar la realidad educativa, regional y particularmente
la lingüística.

\item Proponer metodologías específicas para la enseñanza de la lengua
extranjera, del castellano y del quechua.

\item Planificar la enseñanza de las lenguas en los diferentes niveles de
enseñanza del sistema educativo nacional: inicial, primario, secundario y
universitario.

\item Evaluar, diseñar y/o adaptar material de apoyo para la enseñanza de
lenguas extranjeras.

\end{itemize}

A través de estos objetivos profesionales se desarrolla y facilita capacidades
en los estudiantes que favorecen su desenvolvimiento sin dificultad en el
ámbito laboral. \cite{CMNPZ2014}