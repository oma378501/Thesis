\cleardoublepage
\hypertarget{Ficha Resumen}{}
\bookmark[level=-1,dest=Ficha Resumen]{Ficha Resumen}

\chapter*{Ficha Resumen}

La Carrera de Lingüística Aplicada a la Enseñanza de Lenguas (LAEL), de la
Universidad Mayor de San Simón, elabora material educativo con el objetivo de
brindar recursos, para la enseñanza y fortalecimiento de las lenguas de
francés, ingles y quechua, a toda persona quien tenga el interés de mejorar
su conocimiento respecto a una lengua.

Para lograr la elaboración de los mismos es que, en el año 2014, la Carrera de
LAEL firmó un proyecto de adscripción con la Carrera de Informática. De esta
manera, el trabajo que se debe realizar consiste en que, una parte los
adscritos de LAEL realizan la creación de temas y, posteriormente, la creación
de recursos digitales denominados: \textquotedouble{podcasts educativos}; y por
otra parte, los adscritos de Informática realizaran una plataforma para la
gestión de poscast, subtitulado que consiste en sincronizar la transcripción
con un reproductor de tipo audio.

De modo que, se utiliza suscripción por lenguaje (francés, inglés y quechua) de
interés como mecanismo de para mantener actualizado a un usuario de sistema a
los nuevos podcasts disponibles. 